\section{Introduction, simplices}\label{905}
Topics to be covered:
\begin{enumerate}
    \item Singular homology
    \item CW-complexes
    \item Basics of category theory
    \item Homological algebra
    \item The K\"{u}nneth theorem
    \item UCT, cohomology
    \item Cup and cap products, and
    \item Poincar\'{e} duality.
\end{enumerate}
Examples to keep in mind: $\mathbf{R}^n$. Inside, $S^{n-1}$ being the $(n-1)$-sphere. This is topologized as a subspace of Euclidean space. Taking the quotient by the equivalence relation $x\simeq -x$ gives $\mathbf{RP}^{n-1}$, i.e. the space of lines through the origin in $\mathbf{R}^{n+1}$.

Also, we can look at $V_k(\mathbf{R}^n)$, which is the space of orthonormal $k$-frames (ordered collection of $k$ orthonormal vectors) in $\mathbf{R}^n$, called the Stiefel manifold, topologized as a subspace of $(S^{n-1})^k$. The Grassmannian, $\mathrm{Gr}_k(\mathbf{R}^n)$, the space of $k$-dimensional linear subspaces of $\mathbf{R}^n$. It's a quotient space of $V_k(\mathbf{R}^n)$, since a $k$-frame spans a $k$-dimensional subspace. For example, $\mathbf{Gr}_1(\mathbf{R}^n) = \mathbf{RP}^{n-1}$. These are all \emph{manifolds}.
\begin{definition}
A manifold is a Hausdorff space, locally homeomorphic to $\mathbf{R}^n$.
\end{definition}
All these manifolds are compact except for $\mathbf{R}^n$. Manifolds exhibit a hidden symmetry, captured by Poincar\'{e} duality.

We'll probe general topological spaces through simplices.
\begin{definition}
The standard $n$-simplex $\Delta^n$ is the convex hull of $\{e_0,\cdots,e_n\}$ in $\mathbf{R}^{n+1}$. More explicitly,
$$\Delta^n = \{\sum t_i e_i = 1 : \sum t_i = 1, t_i\geq 0\}\subseteq\mathbf{R}^{n+1}$$
The $t_i$s are called barycentric coordinates.
\end{definition}
Usually we just drop the $e_i$s and just write ``$i$''. There are maps between them, namely inclusions of faces. The map $d^2:\Delta^1\to\Delta^2$ that misses the vertex $i$ (where $0\leq i\leq 2$) is denoted $d^i$\todo{Insert an image here}. They're called ``face inclusions'', and are maps $d^i:\Delta^{n-1}\to\Delta^n$ where $0\leq i\leq n$; they miss the vertex $i$.
\begin{definition}
Let $X$ be any topological space. A singular $n$-simplex in $X$ is a continuous map $\Delta^n\to X$. Denote by $\mathrm{Sin}_n(X)$ as the collection of all $n$-simplices of $X$.
    
    This seems like a ``fairly insane'' thing to do.
\end{definition}
See drawing for a torus\todo{Image me}. The direction of the simplex is like an orientation given by ordering of the indices of $\Delta^n$. The standard notation is $\sigma:\Delta^n\to X$\todo{Depict this whole thing as an image, and remove the sentence when done}.

If $\sigma:\Delta^n\to X$. Find $(n-1)$-simplices by looking faces of simplex; get a map $d_i:\Sin_n(X)\to\Sin_{n-1}(X)$ for $0\leq i\leq n$ by taking $\sigma\mapsto\sigma\circ d^i=: d_i\sigma$, which is the $i$th face of $\sigma$. This carries a lot of information about the space.

Some simplices are particularly interesting. The issue of when boundaries match up is a key aspect of simplices. If $\sigma$ is a simplex that goes around the hole in a torus, then $d_1\sigma = d_0\sigma$, and this means that the ``boundary'' is zero. So we want to understand things like $d_0\sigma - d_1\sigma$, but this isn't even a simplex anymore; we need to take formal sums and differences. We'll therefore consider the abelian group generated by simplices.
\begin{definition}
The abelian group of singular $n$-chains in $X$ is the free abelian group generated by $n$-simplices
$$S_n(X) = \mathbf{Z}\Sin_n(X)$$
    Its elements are finite linear combinations, i.e. formal sums of the form $\sum_{i\in\text{finite set}}a_i\sigma_i$\todo{idk what was meant by ``display'' in the comments} where $a_i\in\mathbf{Z}$. It's a pretty big group. If $n<0$, say that $S_n(X)=0$. Now, define a map $\partial:\Sin_n(X)\to S_{n-1}(X)$ via:
$$\partial\sigma = \sum_{i=0}^n(-1)^i d_i\sigma$$
We'll extend this to $S_n(X) \to S_{n-1}(X)$ by additivity.
\end{definition}
Cycles are chains whose boundary is zero. These are the interesting chains. A more precise definition is the following:
\begin{definition}
An $n$-cycle in $X$ is an $n$-chain $c$ with $\partial c = 0$. Denote $Z_n(X) = \ker(S_n(X)\xrightarrow{\partial}S_{n-1}(X))$.
\end{definition}
For example, with the $\sigma$ on the torus described before, $\partial c = d_0\sigma - d_1\sigma$ and this is zero.
\begin{theorem}
Any boundary is a cycle, i.e., $B_n(X) := \mathrm{Im}(\partial:S_{n+1}(X)\to S_n(X))\subseteq Z_n(X)$.
\end{theorem}
\begin{proof}
    Homework!
\end{proof}
\begin{definition}
The $n$th singular homology group of $X$ is:
    $$ H_n(X) = Z_n(X)/B_n(X) = \frac{\ker(\partial:S_n(X)\to S_{n-1}(X))}{\mathrm{Im}(\partial:S_{n+1}(X)\to S_n(X))}$$
The kernel is a free abelian group, and so is the image because they're both subgroups of free abelian groups; but the quotient isn't necessarily free abelian. The quotient is finitely generated for all the spaces we are talking about although the kernel and image are uncountably generated.
\end{definition}
