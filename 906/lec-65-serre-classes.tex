\section{Serre classes}
The whole course was more or less Serre's PhD thesis.
One thing he did was to explain that you can get qualitative information about $F\to E\to B$ from spectral sequences even if you can't do explicit generators and relations kind of computations.
He did this by describing what are known as \emph{Serre classes}.
\begin{definition}
    A class $\cC$ of abelian groups is a \emph{Serre class} if:
    \begin{enumerate}
	\setcounter{enumi}{0}
	\item $0\in \cC$.
	\item if I have a short exact sequence $0\to A\to B\to C\to 0$, then $A\& C\in \cC$ if and only if $B\in\cC$.
    \end{enumerate}
\end{definition}
Here are some consequences:
a Serre class is closed under isomorphisms (easy).
A Serre class is closed under subobjects and quotients, because for instance $0\to A\hookrightarrow B \to B/A\to 0$ is a short exact sequence.
If $A\to B\to C$ is exact (without the zeros), then if $A,C\in \cC$, then $B\in \cC$ because:
$$
\xymatrix{
    & & & \coker i\ar[d]\\
    & A\ar[r]^i\ar[d] & B\ar[r]^p\ar[ur] & C\\
    0\ar[r] & \ker p\ar[ur] & &
}
$$
\begin{example}
    \begin{enumerate}
	\item $\cC = \{0\}$, and $\cc$ the class of all abelian groups.
	\item $\cC$ being the torsion abelian groups.
	    If I have $0\to A\xrightarrow{i} B\xrightarrow{p} C$; if $A$ and $C$ are torsion, I claim that $B$ is torsion.
	    To see this, let $b\in B$.
	    Then $p(b)$ is killed by $n$, so there exists $a\in A$ such that $i(a) = b$, but $A$ is torsion.
	\item Let $\cP$ be a set of primes.
	    Then define:
	    $$
	    \cC_{\cP} = \{ A : \text{if }p\not\in\cP,\text{ then }p:A\xrightarrow{\simeq} A\text{, i.e., }A \text{ is a } \Z[1/p]\text{-module}\}
	    $$
	    Let $\Z_{(\cP)} = \Z[1/p: p\not\in\cP]\subseteq\QQ$.
	    
	    For instance, if $\cP$ is all primes, then $\cC_\cP$ is all abelian groups.
	    If $\cP$ is all primes other than $\ell$, then $\cC_\cP$ is $\Z[1/\ell]$-modules.
	    You're localizing away from $\ell$.
	    If $\cP = \{\ell\}$, then $\cC_{\{\ell\}} =:\cC_\ell$ is $\Z_{(\ell)}$-modules.
	    If $\cP = \emptyset$, then $\cC_\emptyset$ is all rational vector spaces.
	\item There's a further thing you can do: 
	    if $\cC$ and $\cC^\prime$ are Serre classes, then so is $\cC\cap \cC^\prime$.
	    For instance, $\cC_\text{tors} \cap \cC_\text{fg} = \cC_\text{finite}$.
	    Also, $\cC_p \cap \cC_\text{tors}$ is $p$-torsion abelian groups.
    \end{enumerate}
\end{example}
\begin{remark}
    \begin{enumerate}
	\item If $C_\bullet$ is a chain complex, and $C_n\in \cC$, then $H_n(C_\bullet)\in\cC$.
	\item Suppose $F_\ast A$ is a filtration on an abelian group.
	    If $A\in\cC$, then $\gr_nA\in\cC$ for all $n$.
	    If $F_\ast A$ is finite and $\gr_n A\in\cC$ for all $n$, then $A\in\cC$.
	\item Suppose I have a spectral sequence $\{E_r\}$.
	    Suppose $E^2_{s,t}\in \cC$.
	    Then $E^r_{s,t}\in \cC$ for $r\geq 2$.
	    And so, if $\{E^r\}$ is a right half-plane spectral sequence, then $E^{s+1}_{s,t}\fib E^{s+2}_{s,t}\fib\cdots\fib E^\infty_{s,t}\in\cC$.

	    Thus, if the spectral sequence comes from a filtered complex (and it's bounded below and for all $n$ there exists an $s$ such that $F_s H_n(C) = H_n(C)$, i.e., the homology of the filtration stabilizes), then
	    we found that $E^\infty_{s,t} = \gr_s H_{s+t}(C)$.
	    This means that if the $E^2_{s,t}\in\cC$ for all $s+t = n$, then $H_n(C)\in\cC$. 
    \end{enumerate}
\end{remark}
To apply this to the Serre spectral sequence, we need an additional axiom.
\begin{enumerate}
	\setcounter{enumi}{1}
    \item if $A,B\in\cC$, then so are $A\otimes B$ and $\Tor_1(A,B)$.
\end{enumerate}
All of the examples given above satisfy this.
\begin{notation}
    $f:A\to B$ is said to be a:
    \begin{itemize}
	\item $\cC$-epi if $\coker f\in \cC$.
	\item $\cC$-mono if $\ker f\in\cC$.
	\item $\cC$-iso if both.
    \end{itemize}
\end{notation}
I want to do ... with the Hurewicz theorem.
\begin{prop}
    Let $\pi:E\to B$ be a fibration and $B$ path connected.
    Let $F = \pi^{-1}(\ast)$ be path connected.
    Suppose $\pi_1(B)$ acts trivially on $H_\ast(F)$.
    (This means that we don't have to worry about local coefficients.)

    Let $\cC$ be a Serre class with Axiom 2.
    Let $p\geq 3$ (not a prime, just an index).
    Assume that $H_n(E)\in\cC$ where $1\leq n<p-1$ and
    $H_s(B)\in \cC$ for $1\leq s<p$.
    Then $H_t(F)\in\cC$ for $1\leq t<p-1$.
\end{prop}
\begin{proof}
    It's gonna be a proof by induction, I guess.

    I'll do the case $p=3$ for started.
    We're gonna want to relate the low-dimension homology of these groups.
    What can I say?
    We know that $H_0(E) = \Z$ since it's connected.
    I have $H_1(E)\to H_1(B)$, via $\pi$.
    This is one of the edge homomorphisms, and thus it surjects (no possibility for a differential coming in).
    I now have a map $H_1(F)\to H_1(E)$.
    But I have a possible $d^2:H_2(B)\to H_1(F)$, which is a transgression that gives:
    $$
    H_2(B)\xrightarrow{\partial} H_1(F)\to H_1(E)\to H_1(B)\to 0
    $$
    
    Let me take a step back and say something general.
    You might be interested in knowing when something in $H_n(F)$ maps to zero in $H_n(E)$.
    I.e., what's the kernel of $H_n(F)\to H_n(E)$.
    The sseq gives an obstruction to being an isomorphism.
    The only way that something can be killed by $H_n(F)\to H_n(E)$ is described by:
    $$
    \ker(H_n(F)\to H_n(E)) = \bigcup\left(\img \text{ of }d^r\text{ hitting }E^r_{0,n}\right)
    $$
    You can also say what the cokernel is:
    it's whatever's left in $E^\infty_{s,t}$ with $s+t = n$.
    These obstruct $H_n(F)\to H_n(E)$ from being surjective.
    
    In the same way, I can do this for the base.
    If I have a class in $H_n(E)$, that maps to $H_n(B)$, the question is: what's the image?
    Well, the only obstruction is the possibility is that the element in $H_n(B)$ supports a nonzero differential.
    Thus:
    $$
    \img(H_n(E)\xrightarrow{\pi_\ast} H_n(B)) = \bigcap\left(\ker(d^r:E^r_{r,0}\to\cdots)\right)
    $$
    Again, you can think of the sseq as giving obstructions.
    And also, the obstruction to that map being a monomorphism that might occur in lower filtration along the same total degree line.

    Back to our argument.
    We had the low-dimensional exact sequence:
    $$
    H_2(B)\xrightarrow{\partial} H_1(F)\to H_1(E)\to H_1(B)\to 0
    $$
    Here $p=3$, so we have $H_2(B)\in\cC$ and $H_1(E)\in\cC$.
    Thus $H_1(F)\in\cC$.
    That's the only thing to check when $p=3$.

    Let's do one more case of this induction.
    What does this say?
    Now I'll do $p=4$.
    We're interested in knowing if $E^2_{0,3}\in\cC$.
    There are now two possible differentials!
    I have $H_2(F) = E^2_{0,2}\fib E^3_{0,2}$.
    This quotient comes from $d^2:E^2_{2,1}\to E^2_{0,2}$.
    Now, $d^3:E^3_{3,0}\to E^3_{0,2}$ which gives a surjection $E^3_{0,2}\fib E^4_{0,2}\simeq E^\infty_{0,2}\hookrightarrow H_2(E)$.
    Now, our assumptions were that $E^2_{2,1},E^3_{3,0},H_2(E)\in\cC$.
    Thus $E^3_{0,2}\in\cC$ and so $E^2_{0,2} = H_2(F)\in\cC$.
    Ta-da!
\end{proof}
We're close to doing actual calculations, but I have to talk about the multiplicative structure on the Serre sseq first.
