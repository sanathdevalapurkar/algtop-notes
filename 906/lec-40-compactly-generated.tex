\section{Compactly generated spaces}
A lot of homotopy theory is about loop spaces and mapping spaces.
Standard topology doesn't do very well with mapping spaces, so we will narrate the story of \emph{compactly generated spaces}.
One nice consequence of working with compactly generated spaces is that the category is
Cartesian-closed (a concept to be defined below).

\subsection{CGHW spaces}\label{CGWHspaces}
Some constructions commute for ``categorical reasons''.
For instance, limits commute with limits.
Here is an exercise to convince you of a special case of this.
\begin{exercise}%Exercise 4 from 18.906
    Let $X$ be an object of a category $\cc$.
    The \emph{overcategory} (or the \emph{slice category}) $\cc_{/X}$ has objects
    given by morphisms $p:Y\to X$ in $\cc$, and morphisms given by the obvious
    commutativity condition.
    \begin{enumerate}
	\item Assume that $\cc$ has finite products.
	    What is the left adjoint to the functor $X\times -:\cc\to\cc_{/X}$ that sends
	    $Y$ to the object $X\times Y \xar{\pr_1}X$?
	\item As a consequence of Theorem \ref{adjointslimits}, we find that $X\times -:\cc\to\cc_{/X}$ preserves limits.
	    The composite $\cc\to\cc_{/X}\to\cc$, however, probably does not.
	    \begin{itemize}
		\item What is the limit of a diagram in $\cc_{/X}$?
		\item Let $Y:\cI\to\cc$ be any diagram. Show that
		    $${\lim_{i\in\cI}}^{\cc_{/X}}(X\times Y_i)
		    \simeq X\times {\lim_{i\in\cI}}^\cc Y_i.$$
		    What happens if $\cI$ only has two objects and only identity morphisms?
	    \end{itemize}
    \end{enumerate}
\end{exercise}
However, colimits and limits need not commute!
An example comes from algebra.
The coproduct in the category of commutative rings is the tensor product (exercise!).
But $\left(\lim \Z/p^k\Z\right) \otimes \QQ \simeq \Z_p \otimes \QQ \simeq \QQ_p$ is
clearly not $\lim\left(\Z/p^k\Z\otimes \QQ\right) \simeq \lim 0 \simeq 0$!

We also need not have an isomorphism between
$X\times\colim_{j\in \cJ}Y_j$ and $\colim_{j\in \cJ}(X\times Y_j)$.
One example comes a quotient map $Y\to Z$: 
in general, the induced map $X\times Y\to X\times Z$ is not necessarily another quotient map.
A theorem of Whitehead's says that this problem is rectified if we assume that $X$ is
a compact Hausdorff space.
Unfortunately, a lot of interesting maps are built up from more ``elementary'' maps by such a procedure,
so we would like to repair this problem.
%Why were we talking about colimits? Here's an observation. Suppose $X\to Y$ is a quotient map; then a map $Y\to Z$ is continuous iff the composite $X\to Y\to Z$ is continuous. A quotient map \emph{is} a coequalizer. What I'm saying is, I can find two maps to $X$ such that $Y$ is a coequalizer of $X$. What space are we mapping into $X$? Well, suppose $Z=X/\sim$. If we considered:
%\begin{equation*}
%    \begin{tikzcd}
%	X\times_Z X\ar[r,shift left=.75ex,"\pi_1"]\ar[r,shift right=.75ex,swap,"\pi_2"] & X\ar[r] & Z
%    \end{tikzcd}
%\end{equation*}
%The term here is ``regular epimorphism''.

We cannot simply do this by restricting ourselves to compact Hausdorff spaces:
that's a pretty restrictive condition to place.
Instead (motivated partially by the Yoneda lemma),
we will look at topologies detected by maps from compact Hausdorff spaces.
\begin{definition}
    Let $X$ be a space.
    A subspace $F\subseteq X$ is said to be \emph{compactly closed} if,
    for any map $k:K\to X$ from a compact Hausdorff space $K$, 
    the preimage $k^{-1}(F)\subseteq K$ is closed.
\end{definition}
It is clear that any closed subset is compactly closed, but
there might be compactly closed sets which are not closed in the topology on $X$.
This motivates the definition of a $k$-space:
\begin{definition}
    A topological space $X$ is said to be a \emph{$k$-space}
    if every compactly closed set is closed.
\end{definition}
The $k$ comes either from ``kompact'' and/or Kelly, who was an early topologist
who worked on such foundational topics.

It's clear that $X$ is a $k$-space if and only if the following statement is true:
a map $X\to Y$ is continuous if and only if, for every compact Hausdorff space $K$ and map $k:K\to X$,
the composite $K\to X\to Y$ is continuous.
For instance, compact Hausdorff spaces are $k$-spaces. First countable (so metric spaces) and CW-complexes are also $k$-spaces.

In general, a topological space $X$ need not be a $k$-space.
However, it can be ``$k$-ified'' to obtain another $k$-space denoted $kX$.
The procedure is simple: endow $X$ with the topology consisting of all compactly closed sets.
The reader should check that this is indeed a topology on $X$;
the resulting topological space is denoted $kX$.
This construction immediately implies, for instance, that the identity $kX\to X$ is continuous.

Let $k\Top$ be the category of $k$-spaces.
This is a subcategory of the category of topological spaces, via a functor $i:k\Top\hookrightarrow \Top$.
The process of $k$-ification gives a functor $\Top\to k\Top$, which has the property that:
$$k\Top(X,kY)=\Top(iX,Y).$$
Notice that this is another example of an adjunction!
We can conclude from this that $k(iX\times iY)=X\times^{k\Top}Y$, where $X$ and $Y$ are $k$-spaces.
One can also check that $kiX\simeq X$.

The takeaway is that $k\Top$ has good categorical properties inherited from $\Top$:
it is a complete and cocomplete category.
As we will now explain, this category has more categorical niceness, that does not exist in $\Top$.

\subsection{Mapping spaces}\label{mappingspaces}
Let $X$ and $Y$ be topological spaces.
The set $\Top(X,Y)$ of continuous maps from $X$ to $Y$ admits a topology, namely the compact-open topology.
If $X$ and $Y$ are $k$-spaces, we can make a slight modification:
define a topology on $k\Top(X,Y)$ generated by the sets
$$W(k:K\to X, \text{ open }U\subseteq Y) := \{f:X\to Y: f(k(K))\subseteq U\}.$$
We write $Y^X$ for the $k$-ification of $k\Top(X,Y)$.
\begin{prop}
    \begin{enumerate}
	\item The functor $(k\Top)^{op}\times k\Top\to k\Top$ given by $(X,Y)\to Y^X$ is a functor of both variables.
	\item $e:X\times Z^X\to Z$ given by $(x,f)\mapsto f(x)$ and $i:Y\to (X\times Y)^X$ given by $y\mapsto(x\mapsto(x,y))$ are continuous.
    \end{enumerate}
\end{prop}
\begin{proof}
    The first statement is left as an exercise to the reader.
    For the second statement, see \cite[Proposition 2.11]{StricklandCGWH}.
\end{proof}
As a consequence of this result, we can obtain a very nice adjunction.
Define two maps:
\begin{itemize}
    \item $k\Top(X\times Y,Z)\to k\Top(Y,Z^X)$ via
	$$(f:X\times Y\to Z)\mapsto (Y\xrightarrow{i}(X\times Y)^X\to Z^X).$$
    \item $k\Top(Y,Z^X) \to k\Top(X\times Y,Z)$ via
	$$(f:Y\to Z^X)\mapsto(X\times Y\to X\times Z^X\xrightarrow{e} X).$$
\end{itemize}
By \cite[Proposition 2.12]{StricklandCGWH}, these two maps are continuous inverses, so there is a natural homeomorphism
$$k\Top(X\times Y,Z)\simeq k\Top(Y,Z^X).$$
This motivates the definition of a {Cartesian closed} category.
\begin{definition}
    A category $\cc$ with finite products is said to be
    \emph{Cartesian closed} if, for any object $X$ of $\cc$, the functor $X\times -:\cc\to \cc$ has a right adjoint.
\end{definition}
Our discussion above proves that $k\Top$ is Cartesian closed, while this is not satisfied by $\Top$.
As we will see below, this has very important ramifications for algebraic topology.
\begin{exercise}
    \todo{Insert Exercise 2 from 18.906.}
\end{exercise}
