\section{Simplicial sets}
pset 3 is now due Friday, corrected on Wednesday.

Simplicial sets arose first in algebraic topology, and now they're everywhere.
Really I should be saying ``simplicial things'', or actually simplicial objects.
Folks in 905 know about this, because I forced you through this in the beginning of the course.
\subsection{Review}
$[n]$ is the set $\{0,1,\cdots,n\}$ as a totally ordered set.
I define a category $\Deltab = \{[n]:n\geq 0\}$ with morphisms order preserving maps.
We can define a functor $\Delta:\Deltab\to\Top$ defined via $[n]\mapsto\Delta^n$ which is the convex hull of $\{e_0,\cdots,e_n\}\subseteq \RR^{n+1}$.
This is extremely familiar from simplicial homology.
I have the vertices of these things indexed by elements of $[n]$, so I just extend as an affine map. Namely, $\phi:[m]\to [n]$ gives $\phi:\Delta^m\to\Delta^n$, an affine extension. 
Sometimes $\Delta^n$ is called the standard simplex.

The next thing we did was to take some space $X$ and think about the set of singular $n$-simplices $\Top(\Delta^n,X)$, which gives a contravariant functor $\Deltab^{op}\to\Set$ via $[n]\mapsto\Top(\Delta^n,X)$. This gives the singular simplicial set $\Sin$.
\begin{definition}
    Let $\cc$ be a category. Denote by $s\cc$ the category of simplicial objects in $\cc$, i.e., the category $\Fun(\Deltab^{op},\cc)$.
    Typically we write $X_n = X([n])$, called the $n$-simplices.
\end{definition}
Expanding this out, for all $n\geq 0$, there is $X_n\in\cc$ and maps given by the following. There are maps $d^i:[n]\to [n+1]$ given by omitting $i$ (called coface maps)
and codegeneracy maps $s^i:[n]\to[n-1]$ that's the surjection which repeats $i$.
Any order-preserving map can be written as the composite of these maps, and there are famous relations that these things satisfy.
They generate the category $\Deltab$.
Hence they give maps $d_i:X_{n+1}\to X_n$ and $s_i:X_{n-1}\to X_n$. These are the face and degeneracy maps.
They're called that for a good reason, because in $\Sin(X)$, the face map $d_i\sigma = \sigma\circ d^i$, i.e., the simplex composed with the inclusion of a face.
\begin{example}
    Suppose $\cc$ is a small category, for instance, a group.
    Notice that $[n]$ is a small category, with:
    $$
    [n](i,j) = \begin{cases}
	\{\leq\} & \text{if }i\leq j\\
	\emptyset & \text{else}
    \end{cases}
    $$
    So, I'm entitled to think about $\Fun([n],\cc)$.
    I thus have a simplicial set $N\cc$ whose $n$-simplices are $(N\cc)_n = \Fun([n],\cc)$.
    This is called the \emph{nerve} of $\cc$.
    Interestingly enough $N$ defines a functor $\Cat\to\Set$.
    Thus an $n$-simplex in the nerve is $(n+1)$-objects in $\cc$ (possibly with repetitions) and a chain of $n$ composable morphisms.
    The face maps in the nerve are truncating (at the end) or composing.
    The degeneracy maps just compose with the identity at that vertex.

    For example, $(NG)_n = G^n$ where $G$ is a group regarded as a category.
\end{example}
\subsection{Realization}
We went from spaces to simplicial sets. There's a way to go back.
This was thanks to Milnor. Lucky guy to be able to spell out this construction.

Suppose I have a simplicial set $X$. I'll define $|X|$ as follows:
$$
|X| = \left(\coprod_{n\geq 0}\Delta^n\times X_n\right)/\sim
$$
where $\sim$ is defined as:
$$
\Delta^m\times X_m\ni (v,\phi^\ast x)\sim (\phi_\ast v, x)\in \Delta^n\times X
$$
for all $v\in \Delta^m$ and $x\in X_n$, where $\phi:[m]\to [n]$.
Let's spell this out a little bit.
\begin{example}
    Let's look at $\phi^\ast = d_i:X_{n+1}\to X_n$ and $\phi_\ast = d^i:\Delta^n\to \Delta^{n+1}$.
    The equivalence relation is then saying that if I have $(v,d_ix)\in \Delta^n\times X_n$, that's supposed to be equivalent to $(d^i v, x)\in \Delta^{n+1}\times X_{n+1}$.
    So it's saying: I have the copy of $\Delta^{n+1}$ indexed by $x\in X_{n+1}$, and I have a copy of $\Delta^n$ indexed by $d_i x$. I'm taking the $n$-simplex indexed by $d_i x$ and identifying it with the $i$th face of the $(n+1)$-simplex indexed by $x$.
    It's sticking together simplices as dictated by the simplicial structure on $X$.
\end{example}
There's a similar picture for the degeneracies $s^i$.
For this, $\sim$ is saying that every element of the form $(v,s_ix)$ is already represented by a simplex of lower dimension.
Maybe we should do an example of this.
\begin{example}
    Pick an $n\geq 0$. I'll look at $\Deltab(-,[n])$. This is a simplicial set.
    It's like the most canonical simplicial set of all.
    Let's call $\Deltab(-,[n])$ the ``simplicial $n$-simplex'', sorry about that. This is denoted $\Deltab^n$.
    What's $|\Deltab^n|$?
    This is gonna be:
    $$
    |\Deltab^n| = \coprod_{\text{nondegenerate $k$-simplices in }\Deltab^n}\Delta^k/\sim
    $$
    What are these nondegenerate $k$-simplices in $\Deltab^n$? These are exactly injective order preserving maps $[k]\to[n]$.
    And then these are going to glue them together.

    Suppose $n=3$; what are the injective maps $[k]\to [n]$? First $k\leq 3$. I have the identity.
    Injective maps $[2]\to [3]$ just omit one vertex, so I can put in a face. There are four of them.
    We can do the same thing with $1$-simplices.
    Those are the edges, and same thing for $0$-simplices.
    
    What I've tried to prove is that $|\Deltab^n| \simeq \Delta^n$.
    I've also tried to show that $|X|$ is naturally a CW-complex, with $\mathrm{sk}_n|X| = \left(\coprod_{k\leq n}\Delta^k\times X_k\right)/\sim$.
    These face maps give you the attaching maps.
    This is a very combinatorial way to produce CW-complexes.
\end{example}
Now, we have two functors going back and forth between spaces and simplicial sets?
Do they form an adjoint pair?
Actually this is one of the motivating examples of the theory of adjoint pairs. Namely, I have:
$$
|-|:s\Set\stackrel{\rightarrow}{\leftarrow} \Top:\Sin
$$
Let $X$ be a space.
We have a map $\Delta^n\times\Sin_n(X)\to X$ given by $(v,\sigma)\mapsto \sigma(v)$.
This is a continuous map.
This equivalence relation defining $|\Sin(X)|$ says that the map factors as:
\begin{equation*}
    \xymatrix{
	|\Sin(X)|\ar@{-->}[r] & X\\
	\coprod\Delta^n\times\Sin_n(X)\ar[ur]\ar@{->>}[u] &
    }
\end{equation*}
This is the counit of the adjunction.

Suppose $K\in s\Set$. I want to define $K\to\Sin|K|$.
If I have $x\in K_n$, I want to send it to $\Delta^n\to |K|$ defined via $v\mapsto [(v,x)]$.

This is the beginning of a long story in semi-classical homotopy theory which says that you can take any homotopy theoretic question and reformulate it in simplicial sets.
For instance, you can define homotopy groups in simplicial sets.
If you don't like point-set topology, you can work in simplicial sets, and many people prefer that.

The final thing to say for what happens next is this. If I have a category $\cc$, its realization $|N\cc|$ is a CW-complex, called the \emph{classifying space} of $\cc$.
That's one thing to say.
The other thing to leave you with is this.
This construction of realization -- exactly the same formula makes sense more generally, even if I was considering simplicial \emph{spaces}.
When I take the product, I really mean the product of spaces (in compactly generated spaces).
