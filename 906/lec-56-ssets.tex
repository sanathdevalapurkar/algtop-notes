\section{Simplicial sets}
In order to discuss the simplicial model for classifying spaces of $G$-bundles,
we will embark on a long digression on simplicial sets (which will last for
three sections). We begin with a brief review of some of the theory of
simplicial objects (see also Part \ref{905}).
\subsection{Review}
We denote by $[n]$ the set $\{0,1,\cdots,n\}$, viewed as a totally ordered set.
Define a category $\Deltab$ whose objects are the sets $[n]$ for $n\geq 0$,
with morphisms order preserving maps. There are maps $d^i:[n]\to [n+1]$ given
by omitting $i$ (called coface maps) and codegeneracy maps $s^i:[n]\to[n-1]$
that's the surjection which repeats $i$. As discussed in Exercise
\ref{simplicialidentities}, any order-preserving map can be written as the
composite of these maps, and there are famous relations that these things
satisfy. They generate the category $\Deltab$.

There is a functor $\Delta:\Deltab\to\Top$ defined by sending
$[n]\mapsto\Delta^n$, the standard $n$-simplex. To see that this is a functor,
we need to show that maps $\phi:[n]\to [m]$ induce maps $\Delta^n\to \Delta^m$.
The vertices of $\Delta^n$ are indexed by elements of $[n]$, so we may just
extend $\phi$ as an affine map to a map $\Delta^n\to\Delta^m$.

Let $X$ be a space. The set of singular $n$-simplices $\Top(\Delta^n,X)$
defines the singular simplicial set $\Sin:\Deltab^{op}\to\Set$.
\begin{definition}
    Let $\cc$ be a category. Denote by $s\cc$ the category of simplicial
    objects in $\cc$, i.e., the category $\Fun(\Deltab^{op},\cc)$. We write
    $X_n = X([n])$, called the $n$-simplices.
\end{definition}
Explicitly, this gives an object $X_n\in\cc$ for every $n\geq 0$, as well as
maps $d_i:X_{n+1}\to X_n$ and $s_i:X_{n-1}\to X_n$ given by the face and
degeneracy maps.
\begin{example}
    Suppose $\cc$ is a small category, for instance, a group. Notice that $[n]$
    is a small category, with:
    $$[n](i,j) = \begin{cases}
	\{\leq\} & \text{if }i\leq j\\
	\emptyset & \text{else}.
    \end{cases}$$
    We are therefore entitled to think about $\Fun([n],\cc)$. This begets a
    simplicial set $N\cc$, called the \emph{nerve of $\cc$}, whose
    $n$-simplices are $(N\cc)_n = \Fun([n],\cc)$. Explicitly, an $n$-simplex in
    the nerve is $(n+1)$-objects in $\cc$ (possibly with repetitions) and a
    chain of $n$ composable morphisms. The face maps are given by composition
    (or truncation, at the end of the chain of morphisms). The degeneracy maps
    just compose with the identity at that vertex.
    
    For example, if $G$ is a group regarded as a category, then $(NG)_n = G^n$.
\end{example}
\subsection{Realization}
The functor $\Sin$ transported us from spaces to simplicial sets. Milnor described a way to go the other way. 

Let $X$ be a simplicial set. We define the realization $|X|$ as follows:
$$|X| = \left(\coprod_{n\geq 0}\Delta^n\times X_n\right)/\sim,$$
where $\sim$ is the equivalence relation defined as:
$$
\Delta^m\times X_m\ni (v,\phi^\ast x)\sim (\phi_\ast v, x)\in \Delta^n\times X
$$
for all maps $\phi:[m]\to [n]$ where $v\in \Delta^m$ and $x\in X_n$.
\begin{example}
    The equivalence relation is telling us to glue together simplices as
    dictated by the simplicial structure on $X$. To see this in action, let us
    look at $\phi^\ast = d_i:X_{n+1}\to X_n$ and $\phi_\ast = d^i:\Delta^n\to
    \Delta^{n+1}$. In this case, the equivalence relation then says that
    $(v,d_ix)\in \Delta^n\times X_n$ is equivalent to $(d^i v, x)\in
    \Delta^{n+1}\times X_{n+1}$. In other words: the $n$-simplex indexed by
    $d_i x$ is identified with the $i$th face of the $(n+1)$-simplex indexed by
    $x$.
\end{example}
There's a similar picture for the degeneracies $s^i$, where the equivalence
relation dictates that every element of the form $(v,s_ix)$ is already
represented by a simplex of lower dimension.
\begin{example}
    Let $n\geq 0$, and consider the simplicial set $\Hom_{\Deltab}(-,[n])$.
    This is called the ``simplicial $n$-simplex'', and is commonly denoted
    $\Deltab^n$ for good reason: we have a homeomorphism $|\Deltab^n| \simeq
    \Delta^n$. It is a good exercise to prove this using the explicit
    definition.
%    $$
%    |\Deltab^n| = \coprod_{\text{nondegenerate $k$-simplices in
    %    }\Deltab^n}\Delta^k/\sim
%    $$
%    Suppose $n=3$; what are the injective maps $[k]\to [n]$? First $k\leq 3$. I have the identity.
%    Injective maps $[2]\to [3]$ just omit one vertex, so I can put in a face. There are four of them.
%    We can do the same thing with $1$-simplices.
%    Those are the edges, and same thing for $0$-simplices.
\end{example}
For any simplicial set $X$, the realization $|X|$ is naturally a CW-complex,
with
$$\mathrm{sk}_n|X| = \left(\coprod_{k\leq n}\Delta^k\times X_k\right)/\sim.$$
The face maps give the attaching maps; for more details, see \cite[Proposition
I.2.3]{goerss-jardine}. This is a very combinatorial way to produce
CW-complexes.

The geometric realization functor and the singular simplicial set give two
functors going back and forth between spaces and simplicial sets. It is natural
to ask: do they form an adjoint pair? The answer is yes:
$$
\begin{tikzcd}
    s\Set\ar[r,bend left,"|-|",""{name=A, below}] & \Top\ar[l,bend
    left,"\Sin",""{name=B,above}] \ar[from=A, to=B, symbol=\dashv]
\end{tikzcd}
$$
For instance, let $X$ be a space. There is a continuous map
$\Delta^n\times\Sin_n(X)\to X$ given by $(v,\sigma)\mapsto \sigma(v)$. The
equivalence relation defining $|\Sin(X)|$ says that the map factors through the
dotted map in the following diagram:
\begin{equation*}
    \xymatrix{
	|\Sin(X)|\ar@{-->}[r] & X\\
	\coprod\Delta^n\times\Sin_n(X)\ar[ur]\ar@{->>}[u] &
    }
\end{equation*}
The resulting map is the counit of the adjunction.

Likewise, we can write down the unit of the adjunction: if $K\in s\Set$, the
map $K\to\Sin|K|$ sends $x\in K_n$ to the map $\Delta^n\to |K|$ defined via
$v\mapsto [(v,x)]$.

This is the beginning of a long philosophy in semi-classical homotopy theory,
of taking any homotopy-theoretic question and reformulating it in simplicial
sets. For instance, one can define homotopy groups in simplicial sets. For more
details, see \cite{goerss-jardine}.

We will close this section with a definition that we will discuss in the next
section. Let $\cc$ be a category. From our discussion above, we conclude that
the realization $|N\cc|$ of its nerve is a CW-complex, called the
\emph{classifying space} $B\cc$ of $\cc$; the relation to the notion of
classifying space introduced in \S \ref{grassmannmodel} will be elucidated upon
in a later section.
%That's one thing to say.
%The other thing to leave you with is this.
%This construction of realization -- exactly the same formula makes sense more generally, even if I was considering simplicial \emph{spaces}.
%When I take the product, I really mean the product of spaces (in compactly generated spaces).
