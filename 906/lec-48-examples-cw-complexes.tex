\section{Some examples; CW complexes}
\subsection{Examples}
\begin{enumerate}
    \item Let $E\to B$ be a covering space with $E$ and $B$ connected. The fibers are discrete, so they don't have any higher homotopy. There's only $\pi_0$. The lexseq says that $\pi_n(E)\to \pi_n(B)$ is an isomorphism for $n>1$, and $\pi_1(E)\hookrightarrow \pi_1(B)$, and the subgroup $\pi_1(E)$ classifies the covering space. In general, we know that $\Omega B$ acts on the homotopy fiber $F$. Because $F$ here is discrete, this action factors through $\pi_0(\Omega B)\simeq \pi_1(B)$.

	In particular, $\pi_q(S^n)\simeq \pi_q(\RP^n)$ for $q>1$. Of course, $\pi_1(\RP^n)\simeq \Z/2\Z$. This creates a ton of homology in $\RP^n$ that's not present in the homology of $S^n$. Here's a piece of language.

	\begin{definition}
	    A space is \emph{$n$-connected} if $\pi_i(X) = 0$ for $i\leq n$.
	\end{definition}
	This is not a nonsense definition, although it seems like we need to pick a basepoint -- but $0$-connected means path connected, so we're good. For instance, $1$-connected means simply connected.

	Suppose $E\to B$ is a covering space where $E$ is $n$-connected. Then $\pi_1(B)$ determines the homology $H_i(B)$ in dimensions $i<n$. In particular, $H_i(B) = H_i(\pi_1(B))$, which is the group homology. This is due to Heinz Hopf.
    \item What's a nontrivial action of $\pi_1$ on higher homotopy? For example, $\pi_1$ acting on $\pi_2$. Consider the space $S^1\vee S^2$. What does the universal cover $E$ look like? This just $\RR$, where at every integer point I put a $2$-sphere $S^2$. The universal cover is simply connected, so Hurewicz says that $\pi_2(E)\simeq H_2(E)$. Up to homotopy type, I can collapse the line to a point, so I get a countable bouquet of $2$-spheres. Thus $\pi_2(E)\simeq H_2(E) = \bigoplus^\infty\Z$. But I can say more!

	There's an action of $\pi_1(S^1\vee S^2)$ on $E$. Certainly $\pi_1(S^1\vee S^2) = \Z$. There's an induced action on $E$. All the action does is shift the $2$-spheres on the integer points of $\RR$ (on $E$) to the right by $1$. Thus $\bigoplus^\infty\Z=\Z[\pi_1(B)]$ as a $\Z[\pi_1(B)]$-module (i.e., as a $\Z[\Z]$-module). That's actually the same action of $\pi_1(E)$ on $\pi_2(E)$.	This tells something horrifying. $S^1\vee S^2$ is a very simple $3$-complex, but its homotopy is huge.
    \item $\pi_\ast(X)=?$. Recall our Hopf fibration $S^1\to S^3\to S^2$. This gives a lexseq on homotopy groups. Thus I get $\pi_i(S^3)\xrightarrow{\simeq}\pi_i(S^2)$ for $i>2$ given by $\alpha\mapsto\eta\alpha$. Of course, $\pi_2(S^2) = \Z$. The higher homotopy groups of $S^2$ are exactly the same as the higher homotopy groups of $S^3$! In particular, $\pi_3(S^3)=\Z$ by Hurewicz, so $\pi_3(S^2)\simeq \Z$ generated by $\eta$. There are some low-dimensional calculations that you can make like this.

	Here's another way of thinking of the Hopf fibration. Think of $S^3 = SU(2)$. Then there's the subgroup of $\begin{pmatrix}\lambda & \\ & \lambda^{-1}\end{pmatrix}$, which is $S^1$. This acts on $S^3$ by translation, and the orbit space is $S^2$.

	We'll show later that $\pi_{4n-1}(S^{2n})\otimes\QQ\simeq \QQ$. Other than $\pi_n(S^n)$, a theorem of Serre says that these are the only non-torsion homotopy groups of spheres.
\end{enumerate}
\subsection{Bringing you up-to-speed on CW-complexes}
I want to study a fairly rich example of CW-complexes.
\begin{definition}
    A \emph{relative} CW-complex is a pair $(X,A)$ together with a fitration $A=X_{-1}\subseteq X_0\subseteq X_1\subseteq\cdots\subseteq X$, such that for all $n$, I can build $X_n$ as the pushout:
    $$
    \xymatrix{
	\coprod_{\alpha\in \Sigma_n}S^{n-1}\ar[r]\ar[d]_{\text{attaching maps}} & \coprod_{\alpha\in \Sigma_n}D^n\ar[d]^{\text{characteristic maps}}\\
	X_{n-1}\ar[r] & X_n
    }
    $$
    ($X_n$ is called the $n$-skeleton), and $X=\varinjlim X_n$.
\end{definition}
If $A=\emptyset$, it's not relative anymore -- it's just a CW-complex. (Here $S^{-1} = \emptyset$.) Often, $X$ will be a CW-complex, and $A$ will be a subcomplex.
\begin{enumerate}
    \item If $A$ is Hausdorff, then so is $X$. Actually, last semester I discovered a proof of this in Hatcher.
    \item $A=\emptyset$: then $X$ is compactly generated. This is one of the reasons I wanted to define compactly generated spaces.
    \item If $X$ and $Y$ are both CW-complexes. Then define $(X\times^k Y)_n = \bigcup_{i+j = n}X_i\times Y_j$. This gives a CW-structure on the product $X\times^k Y$.
    \item Any closed smooth manifold admits a CW-structure. 
\end{enumerate}
The fun example is $\CP^n$.
\begin{example}
    $\CP^n$ is a CW-complex, with skeleta $\CP^0\subseteq\CP^1\subseteq\cdots\subseteq \CP^n$. The assertion is that there's a pushout diagram:
    \begin{equation*}
	\xymatrix{
	    \coprod S^{2n-1}\ar[r]\ar[d] & \coprod D^{2n}\ar[d]\\
	    \CP^{n-1}\ar[r] & \CP^n
	    }
    \end{equation*}
    Any complex line through the origin meets the hemisphere defined by $\begin{pmatrix}z_0\\\vdots\\z_n\end{pmatrix}$ with $||z||=1$ and $\Im(z_n) = 0$ and $\Re(z_n)\geq 0$. It meets this hemisphere (which is $D^{2n}$) at one point, unless it's on the equator, in which case we find that the pushout diagram is really:
    \begin{equation*}
	\xymatrix{
	    S^{2n-1}\ar[r]\ar[d] & D^{2n}\ar[d]\\
	    \CP^{n-1}\ar[r] & \CP^n
	    }
    \end{equation*}
\end{example}
I want to give you a more complicated example that's fun and that will come up later.
\begin{example}[Grassmannians]
    Let $V=\RR^n$ or $\cc^n$ or $\HH^n$. Let's suppose $V=\RR^n$ for now. Then $\Gra_k(\RR^n)$ is the collection of $k$-dimensional subspaces of $\RR^n$. One way to do this is by giving a $k\times n$ rank $k$ matrix. Here's some linear algebra.

    The span of rows of $A$ is the row space $V_A$. This is the span of the rows of the reduced reduced echelon form of $A$. An entry is a \emph{pivot} if its leftmost nonzero is in its row. A column is \emph{pivotal} if it contains a pivot. Any matrix is reduced echelon if the $i$th pivotal column is $e_i$.

    For instance, $\Gra_2(\RR^4)$ is as a set the disjoint union of:
    \begin{equation*}
	\begin{pmatrix}& 1 & \\ & & 1\end{pmatrix},\begin{pmatrix}&1&\ast\\&&1\end{pmatrix},\begin{pmatrix}1&\ast&\ast\\&&1\end{pmatrix},\begin{pmatrix}&1&\ast\\&1&\ast\end{pmatrix},\begin{pmatrix}1&\ast&\ast\\&1&\ast\end{pmatrix},\begin{pmatrix}1&\ast&\ast\\1&\ast&\ast\end{pmatrix}
    \end{equation*}
    So, define:
    \begin{definition}
	$\mathrm{sk}_j\Gra_k(\RR^n)$ is $\{V:\text{row ech rep with $\leq j$ free entries}\}$.
    \end{definition}
    The claim is that it's a CW-structure. See Milnor-Stasheff. They don't know it in 18.06, but they're constructing a CW-structure for the Grassmannian.
\end{example}
    The complex Grassmannian has cells in only even dimensions. We now know what the homology of the Grassmannian is, so we actually find Poincare duality if we count the number of cells in this case. (This is visible, for instance, in $\Gra_2(\RR^4)$). By the way, the top-dimensional cell tells us that $\dim\Gra_k(\RR^n) = k(n-k)$.
