\section{... and all the rest}
I'm going to state a few corollaries of the obstruction theory stuff. Davis-Kirk is a great book, and so is Hatcher -- but although it seems to be a chatty book, the proofs are very very dense.
\begin{theorem}[Obstruction theory]
    Let $(X,A)$ be a relative CW-complex, and $Y$ a simple space. The map $[X,Y]\to [A,Y]$ is:
    \begin{enumerate}
	\item is onto if $H^n(X,A;\pi_{n-1}(Y)) = 0$ for all $n\geq 2$.
	\item is one-to-one if $H^n(X,A;\pi_n(Y)) = 0$ for all $n\geq 1$.
    \end{enumerate}
\end{theorem}
I think (1) implies (2), because suppose I have $g_0,g_1:X\to Y$ and a homotopy $h:g_0|_{A}\simeq g_0|_{A}$. Let's apply (1) to $(X\times I,A\times I\cup X\times\partial I)$. Because $A$ is a relative CW-complex, $A\hookrightarrow X$ is a cofibration, as is $A\times I\cup X\times\partial I\to X\times I$. Thus $H^n(X\times I,A\times I\cup X\times\partial I;\pi)\simeq \widetilde{H}^n(X\times I/(A\times I\cup X\times\partial I);\pi) = H^n(\Sigma X/A;\pi)\simeq \widetilde{H}^{n-1}(X/A;\pi)$.

More precisely, if I have $X_{n-1}\to Y$, we get $\theta\in Z^n(X,A;\pi_{n-1}(Y))$, constructed by looking at the attaching map $f_\alpha$ of some $\alpha\in\Sigma_n$, to define $\theta(g)$ via $\theta(g)(\alpha) = [g\circ f_\alpha]$. This captures the obstruction to extending $g$ over $\alpha$. We found that $d\theta = 0$.

\begin{prop}
    Suppose $g:X_{n-1}\to Y$. Then $g|_{X_{n-2}}$ extends to $X_n\to Y$ iff $[\theta(g)] = 0$ in $H^n(X,A;\pi_{n-1}(Y))$.
\end{prop}
Here are some consequences.
\begin{enumerate}
    \item This is called \emph{CW approximation}.
	\begin{theorem}
	    Any space admits a weak equivalence from a CW-complex.
	\end{theorem}
	If you're willing to work up to weak equivalences, then CW-complexes are everything.
    \item If $W$ is a CW-complex and $f:X\to Y$ is a weak equivalence, then $[W,X]\xrightarrow{\simeq}[W,Y]$. This is actually an if and only if statement.
	\begin{corollary}
	    If $X$ and $Y$ are CW-complexes, then a weak equivalence $f:X\to Y$ is a homotopy equivalence.
	\end{corollary}
    \item Let $X$ be path connected. Then there's a space $X_{\leq n}$ and a map $X\to X_{\leq n}$ such that $\pi_i(X_{\geq n}) = 0$ for $i>n$, and $\pi_i(X)\xrightarrow{\simeq}\pi_i(X_{\leq n})$ for $i\leq n$. The pair $(X,X_{\leq n})$ is essentially unique up to homotopy. This space $X_{\leq n}$ is called a \emph{Postnikov section}.
\end{enumerate}
Suppose $A$ is some abelian group. Then there is a space $M(A,n)$ with homology given by:
	\begin{equation*}
	    \widetilde{H}_i(M(A,n)) = \begin{cases}
		A & i = n\\
		0 & i\neq n
	    \end{cases}
	\end{equation*}
	We made this by looking at a free resolution $0\to F_1\to F_0\to A\to 0$ of $A$. Well, $\bigvee S^n\to \bigvee S^n$ realizes the first two maps, and then cone this off to get $M(A,n)$.

	Suppose I apply this ``truncation'' construction to this example; namely, let's look at $M(A,n)_{\leq n}$. What is the $n$-dimensional homotopy of $M(A,n)$? Well, by Hurewicz:
	\begin{equation*}
	    \pi_i(M(A,n)) = \begin{cases}
		0 & i<n\\
		A & i = n\\
		?? & i>n
	    \end{cases}
	\end{equation*}
	Thus when we truncate, we obtain:
	\begin{equation*}
	    \pi_i(M(A,n)_{\leq n}) = \begin{cases}
		A & i = n\\
		0 & i\neq n
	    \end{cases}
	\end{equation*}
	This is a ``designer homotopy type''. This space $M(A,n)_{\leq n}$ is called an \emph{Eilenberg-Maclane space}.

	When $n=1$, $\pi_1$ doesn't have to be abelian. But you can still construct $K(G,1)$. This is called the classifying space of $G$. Next week, I'll talk a bit about classifying spaces. You know examples of this; for instance, if $\Sigma$ is a closed surface not $S^2$ or $\RR^2$, then $\Sigma \simeq K(\pi_1(\Sigma),1)$. Another example is that $S^1\simeq K(\Z,1)$.
	
	Also, $K(\Z,2)\simeq \CP^\infty$. Why's that? We have a fiber sequence $S^1\to S^{2n+1}\to \CP^n$. So the lexseq in homotopy tells us that the homotopy groups of $\CP^n$ are the same as the homotopy groups of $S^1$ until $S^{2n+1}$ starts to interfere. But as $n$ grows, namely, if we take the limit, there's a fibration $S^1\to S^\infty\to \CP^\infty$. But $S^\infty$ is weakly contractible (it has no nonzero homotopy groups), which proves the result. Another example is $K(\Z/2\Z,1)$, which is $\RP^\infty$.

$K(A,n)$ is automatically a simple space. Thus if $k>0$, then $[S^k,K(A,n)] = \pi_k(K(A,n)) = H^n(S^k,A)$. So in fact:
\begin{theorem}[Brown representability]
    If $X$ is a CW-complex, then $[X,K(A,n)] = H^n(X;A)$. 
\end{theorem}
I won't prove this, but they aren't that hard to prove. This puts a different perspective on what cohomology is. We have it captured completely on $K(A,n)$. For instance, any one-dimensional cohomology class determines a map to a circle.

If $X$ is a CW-complex, then $X_{\leq n}$ is a CW-complex also. If it isn't, use cellular approximation and then kill homotopy groups. OK, so let $X$ be path connected. Then $X_{\leq 1} = K(\pi_1(X),1)$. I can then form a tower, which commutes (by uniqueness):
\begin{equation*}
    \xymatrix{
	& \vdots\ar[d] & \cdots\ar[l]\\
	& X_{\leq 3}\ar[d]& K(\pi_3(X),3)\ar[l]\\
	& X_{\leq 2}\ar[d]& K(\pi_2(X),2)\ar[l]\\
	X\ar[r]\ar[ur]\ar[uur]\ar[uuur]& X_{\leq 1}\ar@{=}[r] & K(\pi_1(X),1)
    }
\end{equation*}
Where $K(\pi_n(X),n)\to X_{\leq n}\to X_{\leq n-1}$ is a fiber sequence. This is called a \emph{Postnikov tower}.

Aha, but now I can also do the following.
\begin{equation*}
    \xymatrix{
	\cdots\ar[r]\ar[d] & \cdots\ar[r]\ar@{=}[r] & \vdots\ar[d] & \cdots\ar[l]\\
	X_{>3}\ar[r]\ar[d] & X\ar[r]\ar@{=}[r] & X_{\leq 3}\ar[d]& K(\pi_3(X),3)\ar[l]\\
	X_{>2}\ar[r]\ar[d] & X\ar[r]\ar@{=}[r] & X_{\leq 2}\ar[d]& K(\pi_2(X),2)\ar[l]\\
	X_{>1}\ar[r]\ar[d] & X\ar[r]\ar@{=}[r] & X_{\leq 1}\ar@{=}[r]\ar[d] & K(\pi_1(X),1)\\
	X\ar@{=}[r] & X\ar[r] & \ast
    }
\end{equation*}
Where $X_{>n}\to X\to X_{\leq n}$ is a fiber sequence. See, $X_{>1}$ is the universal cover of $X$. I now have a tower that maps into $X$. The left hand tower is called the \emph{Whitehead tower}, named for George Whitehead.

I can take the fiber of $X_{>1}\to X$, and I get $K(\pi_1(X),0)$. We can continue this to get:
\begin{equation*}
    \xymatrix{
	\cdots\ar[r] & \cdots\ar[r]\ar[d] & \cdots\ar[r]\ar@{=}[d] & \vdots\ar[d] & \cdots\ar[l]\\
	K(\pi_3(X),2)\ar[r] & X_{>3}\ar[r]\ar[d] & X\ar[r]\ar@{=}[d] & X_{\leq 3}\ar[d]& K(\pi_3(X),3)\ar[l]\\
	K(\pi_2(X),1)\ar[r] & X_{>2}\ar[r]\ar[d] & X\ar[r]\ar@{=}[d] & X_{\leq 2}\ar[d]& K(\pi_2(X),2)\ar[l]\\
	K(\pi_1(X),0)\ar[r] & X_{>1}\ar[r]\ar[d] & X\ar[r]\ar@{=}[d] & X_{\leq 1}\ar@{=}[r]\ar[d] & K(\pi_1(X),1)\\
	& X\ar@{=}[r] & X\ar[r] & \ast
    }
\end{equation*}
By the fiber sequence $\Omega X\to PX\to X$ with $PX\simeq \ast$, we find that $\Omega K(\pi,n)\simeq K(\pi,n-1)$. Note that these Eilenberg-Maclane spaces are unique up to homotopy.

I've laid out the rest of this basic homotopy theory story, and the proofs from a cellular point of view are annoying, but they're not difficult. It's kind of implicit, though, so you can't really use attaching cells to compute, say the cohomology of Eilenberg-Maclane spaces. That's a harder problem.

These constructions go back to the 50's, and they had voluminous computations in low dimensions. One day in 1950, they got a postcard from Serre, who said, ``here's a computation you might be interested in: $H^{23}(K(\Z,14)) = ...$''. Of course, Serre and Cartan had a different approach, that was much more effective. They observed that the fact $\Omega K(\pi,n)\simeq K(\pi,n-1)$ wasn't perceived to be useful by Eilenberg and Maclane. They didn't think about fiber sequences. Serre and Cartan did this by means of a spectral sequence. We'll do that later in the course.
