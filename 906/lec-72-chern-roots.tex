\section{Chern roots, Stiefel-Whitney and Pontryagin classes}
Pset 6 is due Monday.
Don't worry about the functoriality of $H_s(K(A,n))$.

Let me just repeat what I was rushing through at the end on Wednesday.
I wrote slightly incorrect things.
Let's look at $\cc^n$.
Of course, $U(n)$ acts on this, and preserves the inner product.
Recall that $Fr(\cc^n)$ is the space of orthonormal bases of $\cc^n$.
Thus there's an action of $U(n)$ on this, and this is simply transitive, i.e., it's free and transitive.
Some would say that this is a ``torsor for $U(n)$''.
There's a basepoint in here, namely the standard basis, but you may forget about that.

Now I can make various constructions.
For instance, think about $\Fl(\cc^n)$, which is the space of ordered orthonormal sets of $n$ lines in $\cc^n$.
There's lots of isotropy now.
The isotropy group of the $U(n)$ action on the lines through the standard basis is now $T_n=U(1)^n$, i.e., diagonal matrices with roots of unity on the diagonal.
This is called the \emph{(complete) flag manifold}.

Another thing you could do as well -- you can forget the ordering.
This doesn't really have a name, I guess, but what I'm doing is considering $\Fl(\cc^n)/\Sigma_n$, which is the space of unordered orthonormal lines in $\cc^n$.
This is a homogeneous space, too, and it's $U(n)$ quotiented by permutation matrices where $1$ is any complex number of norm $1$.
For instance, if $N_{U(n)}(T^n) = \Sigma_n\cdot T^n$ is the normalizer, then $\Fl(\cc^n)/\Sigma_n = U(n)/N(T^n)$.
I can consider the cofiber sequence $T^n\hookrightarrow \Sigma_n\cdot T^n \to W_n$, and the $W_n$ is called the Weyl group.
Something nice here is that
\begin{equation*}
    \xymatrix{
	T^n\ar@{^(->}[r] & \Sigma_n\cdot T^n\ar[r] & W_n\\
	& \Sigma_n\ar@{^(->}[u]\ar[ur]^\simeq & 
    }
\end{equation*}
To get projective space I consider the space of splittings of $\cc^n$ into a line and its orthogonal complement -- but that's just giving a line because the metric determines the orthogonal complement.

Consider a complex $n$-plane bundle $\xi^n\downarrow X$, with a metric.
Then $\Fr(\xi^n)\to X$ is a principal $U(n)$-bundle over $X$.
I can form $E(\xi) = \Fr(\xi)\times_{U(n)}\cc^n$, and $\PP(\xi) = \Fr(\xi)\times_{U(n)} U(n)/(U(1)\times U(n-1))$.
Note that $U(n)/(U(1)\times U(n-1)) = \CP^{n-1}$.
I can also consider $\Fl(\xi) = \Fr(\xi)\times_{U(n} U(n)/T^n$.

We found that $H^\ast(X) \hookrightarrow H^\ast(\Fl(\xi))$.
This is the splitting principle.

Let's look at the universal case $\xi_n\downarrow BU(n)$.
So, $\Fl(\xi_n) = EU(n)\times_{U(n)} U(n)/T^n$, so this is $BT^n$.
It maps down to $BU(n)$, and the map is the one induced by the inclusion $T^n\hookrightarrow U(n)$.
So, for any ring, we find that $H^\ast(BU(n))\hookrightarrow H^\ast(BT^n)$.
We know that 
$$
H^\ast(BT^n) = \Z[t_1,\cdots,t_n]
$$
where $t_i = e(\lambda_i)$ and $|t_i| = 2$, where $\lambda_i = pr_i^\ast\lambda$ and $\lambda\downarrow \CP^\infty$ is the universal line bundle.
What is the image of this map $H^\ast(BU(n))\hookrightarrow H^\ast(BT^n)$?

I have the symmetric group sitting inside of $U(n)$, and hence it acts by conjugation on $U(n)$.
Then this action stabilizes this subgroup $T^n$.
By naturality, it acts on the classifying spaces.
Since $\Sigma_n$ acts by conjugation on $U(n)$, it acts on $BU(n)$ in a way that's homotopic to the identity.
However, each $\sigma\in \Sigma_n$ permutes the factors in $BT^n = (\CP^\infty)^n$.
This tells us that $H^\ast(BU(n);R) \hookrightarrow H^\ast(BT^n;R)^{\Sigma_n}$.
\begin{theorem}[Algebra]
    If you take $R[t_1,\cdots,t_n]$ and let $\Sigma_n$ act by permuting the generators, then
    $$
    R[t_1,\cdots,t_n]^{\Sigma_n} = R[\sigma_1^{(n)},\cdots,\sigma_n^{(n)}]
    $$
    where the $\sigma_i$ are the \emph{elementary symmetric polynomials}, defined via
    $$
    \prod^n_{i=1}(x-t_i) = \sum^n_{j=0} \sigma_i^{(n)}x^{n-i}
    $$
\end{theorem}
For instance, $\sigma_1^{(n)} = -\sum t_i$, and $\sigma_n^{(n)} = (-1)^n\prod t_i$.
In between, they're homogeneous polynomials that probably are familiar to you.

You have to be a little careful at this point.
If I grade $|t_i| = 2$, and we have that $|\sigma_i^{(n)}| = 2i$.
So, this ring $H^\ast(BT^n)^{\Sigma_n}$ has the same size as $H^\ast(BU(n))$.
They have exactly the same Poincar\'{e} series.
You could imagine that maybe it embeds as a lattice inside the integers?

Let's see.
Suppose I have finitely generated abelian groups $M\hookrightarrow N$, with quotient $Q$.
Suppose we know that after tensoring with any field, I get an isomorphism.
Note that this is true in our case, since I can count dimensions.
If $Q\otimes k = 0$, then $Q = 0$.
This is because if I tensor it with $\QQ$, I kill all torsion, so if I get zero, then it's torsion.
If I tensor it with $\FF_p$ and it's zero, then it has no $p$-component.
Thus $M\simeq N$.
In particular, we find that
$$
H^\ast(BU(n);R) \xrightarrow{\simeq} H^\ast(BT^n;R)^{\Sigma_n} = R[\sigma_1^{(n)},\cdots,\sigma_n^{(n)}]
$$
Let's look at the compatibility when I change $n$ for these things.
I can map $R[t_1,\cdots,t_n] \to R[t_1,\cdots,t_{n-1}]$ by sending $t_n\mapsto 0$ and $t_i\mapsto t_i$ for $i\neq n$.
Of course, I can't say that this is equivariant with respect to $\Sigma_n$.
But I can say that it's invariant with respect to an action of $\Sigma_{n-1}$ on $R[t_1,\cdots,t_n]$, where $\Sigma_{n-1}\hookrightarrow \Sigma_n$ as the isotropy group of $n\in\{1,\cdots,n\}$.
So the full invariant sit inside the $\Sigma_{n-1}$ invariants, and hence I get a map
$$
R[t_1,\cdots,t_n]^{\Sigma_n} \to R[t_1,\cdots,t_n]^{\Sigma_{n-1}} \to R[t_1,\cdots,t_{n-1}]^{\Sigma_{n-1}}
$$
If you look at the formula that I wrote before, then you find that for $i<n$, then $\sigma_i^{(n)} \to \sigma_i^{(n-1)}$ and $\sigma_n^{(n)} \mapsto 0$.
\subsection{Where do the Chern classes go?}
To do this, we're going to have to understand the multiplicativity of the Chern class.
This is simple, but I'll spend time on it.
Suppose $\xi^p\downarrow X,\eta^q\downarrow Y$ are oriented real vector bundles.
I can consider $\xi\times\eta\downarrow X\times Y$, which is another oriented real vector bundle.
I can give this an orientation by picking oriented bases for $\xi$ and $\eta$.
There are lot of choices, but I'll put the first one first and the second one second.
I claim that
$$
e(\xi\times\eta) = e(\xi)\times e(\eta) \in H^{p+q}(X\times Y)
$$
We can consider $R$-oriented things, but it's either $\Z/2$-oriented or $\Z$-oriented.
Why is it true?
If I want to compute $D(\xi\times\eta)$, well, it's homeomorphic to $D(\xi)\times D(\eta)$.
If I want to go to the sphere bundle -- oh, really I should say that
$$
u_{\xi\times \eta} = u_\xi\times u_\eta\in H^{p+q}(\Th(\xi)\times\Th(\eta))
$$
Anyway, we find that $S(\xi\times\eta) = D(\xi)\times S(\eta)\cup S(\xi)\times D(\eta)$.
There's a relative K\"unneth formula that tells you that
$$
H^\ast(D(\xi\times\eta),S(\xi\times\eta)) \leftarrow H^\ast(D(\xi),S(\xi))\otimes H^\ast(D(\eta),S(\eta))
$$
and $u_\xi\otimes u_\eta\mapsto u_{\xi\times\eta}$ because it's true on each fiber.
By restricting to the base, you get multiplicativity for the Euler class.

If I let $X=Y$, and $\Delta:X\to X\times X$, then the cross product pulls back to the cup product and direct product pulls back to the Whitney sum.
So, you get that
$$e(\xi\oplus\eta) = e(\xi)\cup e(\eta)$$
What does that say about Chern classes?

If $\xi^n\downarrow X$ is an $n$-dimensional complex vector bundle, then we defined\footnote{Well a complex bundle doesn't really have an orientation but its underlying oriented real vector bundle does.}
$$
c_n(\xi) = (-1)^n e(\xi_\RR)
$$
We're interested in $H^{2n}(BU(n)) \to H^{2n}(BT^n)^{\Sigma_n}$.
I want to know where $c_n(\xi_n)$ goes.
This is $(-1)^ne(\xi)$.
Where does the Euler class go?
Where does $\xi$ come from?

I've got $f:BT^n\to BU(n)$, and I have $\xi_n\downarrow BU(n)$.
Thus, by construction, $f^\ast\xi_n = \lambda_1\oplus\cdots\oplus \lambda_n$.
Thus, $(-1)^ne(\xi)\mapsto (-1)^n e(\lambda_1\oplus\cdots\oplus \lambda_n)$.
We computed that the right hand side (by multiplicativity) maps to $(-1)^nt_1\cdots t_n$.
This is precisely $\sigma_n^{(n)}$!
Thus the top Chern class goes to $\sigma_n^{(n)}$.

Now, I have:
\begin{equation*}
    \xymatrix{
	H^\ast(BU(n)) \ar[r] \ar[d] & H^\ast(BT^n)^{\Sigma_n}\ar[d]\\
	H^\ast(BU(n-1)) \ar[r] & H^\ast(BT^{n-1})^{\Sigma_{n-1}}
    }
\end{equation*}
The diagram commutes, but you have to check how maximal tori sit in these spaces.
We know that going from top left to bottom left to bottom right runs via $c_i\mapsto c_i\mapsto \sigma_i^{(n-1)}$, for $i<n$.
By commutativity, $c_i^{(i)}\mapsto \sigma_i^{(i)}$.
(Note that we're inducting here.)
\subsection{Proving the Whitney sum formula}
Whitney called this the hardest theorem he ever proved.
This formula relates characteristic classes.
Well by the stuff we did above, it's the same as this claim:
$$
\sigma^{(p+q)}_k = \sum_{i+j=k}\sigma_i^{(p)}\cdot\sigma_j^{(q)}
$$
This is happening in $\Z[t_1,\cdots,t_p,t_{p+1},\cdots,t_{p+q}]$.
Here $\sigma_i^{(p)}$ is thought of as a polynomial in $t_1,\cdots,t_p$, while $\sigma_i^{(q)}$ is a polynomial in $t_{p+1},\cdots,t_{p+q}$.
Let's check this.
Recall that
\begin{align*}
    \sum_{k=0}^{p+q} \sigma_k^{(p+q)}x^{p+q-k} & = \prod^{p+q}_{i=1}(x-t_i)\\
    & = \prod^p_{i=1}(x-t_i)\cdot\prod^{p+q}_{j=p+1}(x-t_j)\\
    & = \left(\sum^p_{i=0}\sigma^{(p)}_i x^{p-i}\right)\left(\sum^q_{j=0}\sigma_j^{(p)} x^{q-j}\right)\\
    & = \sum^{p+q}_{k=0}\left(\sum_{i+j=k}\sigma_i^{(p)}\sigma_j^{(q)}\right)x^{p+q-k}
\end{align*}
Now compare coefficients.
