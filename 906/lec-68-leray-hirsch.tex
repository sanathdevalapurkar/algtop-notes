\section{Dress spectral sequence, Leray-Hirsch}
I think I have to be doing something tomorrow, so no office hours then.
The new pset is up, and there'll be one more problem up.
There are two more things about spectral sequences, and specifically the multiplicative structure, that I have to tell you about.
The construction of the Serre sseq isn't the one that we gave.
He did stuff with simplicial homology, but as you painfully figured out, $\Delta^s\times\Delta^t$ isn't another simplex.
Serre's solution was to not use simplices, but to use cubes.
He defined a new kind of homology using the $n$-cube.
It's more complicated and unpleasant, but he worked it out.
\subsection{Dress' sseq}
Dress made the following variation on this idea, which I think is rather beautiful.
We have a trivial fiber bundle $\Delta^t \to \Delta^s\times\Delta^t\to \Delta^s$.
Let's do with this what we did with homology in the first place.
Dress started with some map $\pi:E\to B$ (not necessarily a fibration), and he thought about the set of maps from $\Delta^s\times\Delta^t\to \Delta^s$ to $\pi:E\to B$.
This set is denoted $\Sin_{s,t}(\pi)$.
This forgets down to $S_s(B)$.
Altogether, this $\Sin_{\ast,\ast}(\pi)$ is a functor $\Deltab^{op}\times\Deltab^{op}\to \set$, forming a ``bisimplicial set''.

The next thing we did was to take the free $R$-module, to get a bisimplicial $R$-module $R\Sin_{\ast,\ast}(\pi)$.
We then passed to chain complexes by forming the alternating sum.
We can do this in two directions here!
(The $s$ is horizontal and $t$ is vertical.)
This gives us a double complex.
We now get a spectral sequence!
I hope it doesn't come as a surprise that you can compute the horizontal -- you can compute the vertical differential first, and then taking the horizontal differential gives the homology of $B$ with coefficients in something.
Oh actually, the totalization $tR\Sin_{\ast,\ast}(\pi) \simeq R\Sin_\ast(E) = S_\ast(E)$.
We'll have
$$
E^2_{s,t} = H_s(B;\text{crazy generalized coefficients}) \Rightarrow H_{s+t}(E)
$$
These coefficients may not even be local since I didn't put any assumptions on $\pi$!
This is like the ``Leray'' sseq, set up without sheaf theory.
If $\pi$ is a fibration, then those crazy generalized coefficients is the local system given by the homology of the fibers.
This gives the Serre sseq.

This has the virtue of being completely natural.
Another virtue is that I can form $\Hom(-,R)$, and this gives rise to a multiplicative double complex.
Remember that the cochains on a space form a DGA, and that's where the cup product comes from.
The same story puts a bigraded multiplication on this double complex, and that's true \emph{on the nose}.
That gives rise a multiplicative cohomology sseq.

This is very nice, but the only drawback is that the paper is in German.
That was item one in my agenda.
\subsection{Leray-Hirsch}
This tells you condition under which you can compute the cohomology of a total space.
Anyway.
We'll see.

Let's suppose I have a fibration $\pi:E\to B$.
For simplicity suppose that $B$ is path connected, so that gives meaning to the fiber $F$ which we'll also assume to be path-connected.
All cohomology is with coefficients in a ring $R$.
I have a sseq
$$
E_2^{s,t} = H^s(B;\underline{H^t F}) \Rightarrow H^{s+t}(E)
$$
If you want assume that $\pi_1(B)$ acts trivially so that that cohomology in local coefficients is just cohomology with coefficients in $H^\ast F$.
I have an algebra map $\pi^\ast:H^\ast(B)\to H^\ast(E)$, making $H^\ast(E)$ into a module over $H^\ast(B)$.
We have $E^{\ast,t}_2 = H^\ast(B;H^t(F))$, and this is a $H^\ast(B)$-module.
That's part of the multiplicative structure, since $E_2^{\ast,0} = H^\ast B$.
This row acts on every other row by that module structure.

Everything in the bottom row is a permanent cycle, i.e., survives to the $E_\infty$-page.
In other words
$$
H^\ast(B) = E^{\ast,0}_2 \fib E^{\ast,0}_3 \fib \cdots \fib E^{\ast,0}_\infty
$$
Each one of these surjections is an algebra map.

What the multiplicative structure is telling us is that $E^{\ast,0}_r$ is a graded algebra acting on $E^{\ast,t}_r$.
Thus, $E^{\ast,t}_\infty$ is a module for $H^\ast(B)$.

Really I should be saying that it's a module for $H^\ast(B;\underline{H^0(F)})$.
Can I guarantee that the $\pi_1(B)$-action on $F$ is trivial.
We know that $F\to \ast$ induces an iso on $H^0$ (that's part of being path-connected).
So if you have a fibration whose fiber is a point, there's no possibility for an action.
This fibration looks the same as far as $H^0$ of the fiber is concerned.
Thus the $\pi_1(B)$-action is trivial on $H^0(F)$, so saying that it's a $H^\ast(B)$-module is fine.

Where were we?
We have module structures all over the place.
In particular, we know that $H^\ast(E)$ is a module over $H^\ast(B)$ as we saw, and also $E^{\ast,t}_\infty$ is a $H^\ast(B)$-module.
These better be compatible!

Define an increasing filtration on $H^\ast(E)$ via $F_t H^n(E) = F^{n-t} H^n(E)$.
For instance, $F_0 H^n(E) = F^n H^n(E)$.
What is that?
In our picture, we have the associated quotients along the diagonal on $E^{s,t}_\infty$ given by $s+t = n$.
In the end, since we know that $F^{n+1} H^n(E) = 0$, it follows that
$$F_0 H^n(E) = F^n H^n(E) = E^{n,0}_\infty = \img(\pi^\ast:H^n(B)\to H^n(E))$$
With respect to this filtration, we have
$$
\gr_t H^\ast(E) = E^{\ast,t}_\infty
$$
I learnt this idea from Dan Quillen.
It's a great idea.
This increasing filtration $F_\ast H^\ast(E)$ is a filtration by $H^\ast(B)$-modules, and $\gr_t H^\ast(E) = E^{\ast,t}_\infty$ is true as $H^\ast B$-modules.
It's exhaustive and bounded below.

This is a great perspective.
Let's use it for something.
Let me give you the Leray-Hirsch theorem.
\begin{theorem}[Leray-Hirsch]
    Let $\pi:E\to B$.
    \begin{enumerate}
	\item Suppose $B$ and $F$ are path-connected.
	\item Suppose that $H^t(F)$ is free\footnote{Everything is coefficients in $R$} of finite rank as a $R$-module.
	\item Also suppose that $H^\ast(E)\fib H^\ast(F)$.
    That's a big assumption; it's dual is saying that the homology of the fiber injects into the homology of $E$.
    This is called ``totally non-homologous to zero'' -- this is a great phrase, I don't know who invented it.
    \end{enumerate}
    Pick an $R$-linear surjection $\sigma:H^\ast(F)\to H^\ast(E)$; this defines a map $\overline{\sigma}:H^\ast(B)\otimes_R H^\ast(F)\to H^\ast(E)$ via $\overline{\sigma}(x\otimes y) = \pi^\ast(x)\cup \sigma(y)$.
    This is the $H^\ast(B)$-linear extension.
    Then $\overline{\sigma}$ is an isomorphism.
\end{theorem}
\begin{remark}
    It's not natural since it depends on the choise of $\sigma$.
    It tells you that $H^\ast(E)$ is free as a $H^\ast(B)$-module.
    That's a good thing.
\end{remark}
\begin{proof}
    I'm going to use our Serre sseq
    $$
    E^{s,t}_2 = H^s(B;\underline{H^t F}) \Rightarrow H^{s+t}(E)
    $$
    Our map $H^\ast(E)\to H^\ast(F)$ is an edge homomorphism in the sseq, which means that it factors as $H^\ast(E)\to E^{0,\ast}_2 = H^0(B;\underline{H^\ast(F)}) \subseteq H^\ast(F)$.
    Since $H^\ast(E)\to H^\ast(F)$, we have $H^0(B;\underline{H^\ast(F)}) \simeq H^\ast(F)$.
    Thus the $\pi_1(B)$-action on $F$ is trivial.
    \begin{question}
	What's this arrow $H^\ast(E)\to E^{0,\ast}_2$?
	We have a map $H^\ast(E)\to H^\ast(E)/F^1 = E^{0,\ast}_\infty$.
	This includes into $E^{0,\ast}_2$.
    \end{question}
    Now you know that the $E_2$-term is $H^s(B;H^t(F))$.
    By our assumption on $H^\ast(F)$, this is $H^s(B)\otimes_R H^t(F)$, as algebras.
    What do the differentials look like?
    I can't have differentials coming off of the fiber, because if I did then the restriction map to the fiber wouldn't be surjective, i.e., that $d_r|_{E^{0,\infty}_r} = 0$.
    The differentials on the base are of course zero.
    This proves that $d_r$ is zero on every page by the algebra structure!
    This means that $E_\infty = E_2$, i.e., $E_\infty^{\ast,t} = H^\ast(B)\otimes H^t(F)$.

    Now I can appeal to the filtration stuff that I was talking about, so that $E^{\ast,t}_\infty = \gr_t H^\ast(E)$.
    Let's filter $H^\ast(B)\otimes H^\ast(F)$ by the degree in $H^\ast(F)$, i.e., $F_q = \bigoplus_{t\leq q} H^\ast(B)\otimes H^t(F)$.
    The map $\overline{\sigma}:H^\ast(B)\otimes H^\ast(F)\to H^\ast(E)$ is filtration preserving, and it's an isomorphism on the associated graded.
    This is the identification $H^\ast(B)\otimes H^t(F) = E^{\ast,t}_\infty = \gr_t H^\ast(E)$.
    Since the filtrations are exhaustive and bounded below, we conclude that $\overline{\sigma}$ itself is an isomorphism.
\end{proof}
