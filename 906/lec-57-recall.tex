\section{Remember what we have}
We spent the first 25 minutes discussing one of the problems in the pset.

A great reference for simplicial sets is Goerss-Jardine, but it's about 500 pages long.
Another reference is Weibel's homological algebra.
And also May's simplicial objects book.

Remember that $\Deltab\ni[n]$ for $n\geq 0$ where $[n] = \{0,\cdots,n\}$.
A simplicial object is a functor $X:\Deltab^{op}\to\cc$.
Remember that:
$$
|X| = \coprod_{n\geq 0}\Delta^n\times X_n/\sim
$$
Remember that $s^i:[n]\to [n+1]$ repeats $i$. Part of this relation $\sim$ is that $(v,s_i x)\sim (s^i v,x)$.
That means that $x\in X_{n-1}$ and $v\in\Deltab^n$.
This tells us that $\img(s_i:X_{n-1}\to X_{n})$ indexes cells that lie in the $(n-1)$-skeleton $\mathrm{sk}_{n-1}|X| = \coprod_{k\leq n-1}\Delta^k\times X_k/\sim$.

This gives two functors
$$
|-|:s\Set\stackrel{\rightarrow}{\leftarrow} \Top:\Sin
$$
and I defined the unit and counit last time.
\begin{theorem}[Milnor]
    The map $|\Sin(X)|\to X$ is a weak equivalence.
\end{theorem}
This is a functorial CW-approximation, but it's rather large.
It's not at all minimal.
It's some uncountable CW-complex.
\begin{theorem}[You did this]
    There is a homeomorphism $|X\times Y|\xrightarrow{\simeq} |X|\times |Y|$ (product in $k$-spaces).
    The product of two simplicial sets is defined as $(X\times Y)_n:=X_n\times Y_n$.
\end{theorem}
Let $\Cat$ be the category of small categories and functors. The nerve gives a functor:
$$
\Cat\xrightarrow{N}s\Set
$$
\begin{definition}
    In $\Cat$, define $C\times D$ as the category whose objects are pairs of object of $C$ and $D$ and morphisms are pairs of morphisms.
\end{definition}
Then $N(\cc\times\cd) = N\cc\times N\cd$.
This is pretty clear.
Then, I defined $B\cc:=|N\cc|$.
\begin{theorem}
    The natural map $B(\cc\times\cd)\xrightarrow{\simeq}B\cc\times B\cd$ is a homeomorphism.
\end{theorem}
I remember trying to explain to you that there's a simplicial simplex $\Deltab^n = \Deltab(-,[n])$, and trying to convince you that $|\Deltab^n| = \Delta^n$.

Here's another observation about $\Cat$: it's Cartesian closed. There's that word again. I mean, it's got products, and it's got ``function objects'' as well.
If I have $\cd^\cc$, this is a category, and it's the category of functors $\Fun(\cc,\cd)$ and natural transformations.
I leave it to you to verify that this is really the right adjoint of the product.

For instance, here's a category: $[1]$. A functor $[1]\to\cc$ is just an arrow in $\cc$.
So, a functor $[1]\to\cd^\cc$ is a natural transformation between two functors $f_0$ and $f_1$.
This is the same thing as a functor $\cc\times[1]\to\cd$, and in the product category $\cc\times[1]$ is a pair of objects of $\cc$ and a map between them.
So, we're good.

Now, $B([1]\times \cc)\simeq B[1]\times B\cc\to B\cd$. But $B[1] = \Delta^1$.
In other words, this is a homotopy of maps $Bf_0,Bf_1:B\cc\to B\cd$!
\begin{lemma}
    A natural transformation $\theta:f_0\to f_1$ where $f_0,f_1:\cc\to\cd$ induces a homotopy $Bf_0\sim Bf_1:B\cc\to B\cd$.
\end{lemma}
There's something interesting about this.
This notion of a homotopy is \emph{reversible}, but that's not at all true about natural transformations.
One of the conclusions is, therefore, that $B$ forgets the ``polarity'' in $\Cat$.

For example, suppose $L\dashv R$ where $L:\cc\to \cd$; then we have a unit $1_\cc\to RL$ and a counit $LR\to 1_\cd$.
When I apply classifying spaces, I get induced maps $B\cc\to B\cd$ and $B\cd\to B\cc$.
Thus, $B$ of the unit and counit are homotopy equivalences that are inverse up to homotopy.
So if you have two categories that are related by any adjoint pair, then they're homotopy equivalence.

Here's another example. We have the category $[0]$. What is an adjoint pair $L\dashv R$ where $L:[0]\to \cd$? Well $L$ determines an object $\ast$ of $\cd$. But the composition $LR(d)\to d$.
But $L$ of anything is just $\ast$. So this is a unique morphism $\ast\to X$.
It's unique since $\cd(\ast,X) = \cc(0,0) = 0$.
An adjoint pair like that therefore specifies an initial object of $\cd$.

The same way, if I had $L\dashv R$ where $L:\cd\to [0]$, we get a terminal object of $\cd$.
But look, if $\cd$ is any category with a terminal (or initial) object, then $B\cd$ is contractible!

Oh my god, look what time it is! I was just getting warmed up.
