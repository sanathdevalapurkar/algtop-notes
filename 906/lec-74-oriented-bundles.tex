\section{Oriented bundles, Pontryagin classes, Signature theorem}
We have a pullback diagram
$$
\xymatrix{
    BSO(n)\ar[r]\ar[d]^{\text{double cover}} & S^\infty\ar[d]\\
    BO(n)\ar[r]_{w_1} & B\Z/2\Z
}
$$
The bottom map is exactly the element $w_1\in H^1(BO(n);\FF_2)$. It follows
that a vector bundle $\xi\downarrow X$ represented by a map $f:X\to BO(n)$ is
orientable iff $w_1(\xi) = f^\ast(w_1) = 0$, since this is equivalent to the
existence of a factorization:
$$
\xymatrix{
    & BSO(n)\ar[r]\ar[d] & S^\infty\ar[d]\\
    X\ar@{-->}[ur]\ar[r]_{\xi} & BO(n)\ar[r]_{w_1} & B\Z/2\Z
}
$$
The fiber sequence $BSO(n)\to BO(n)\to \RP^\infty$ comes from a fiber sequence
$SO(n)\to O(n)\to\Z/2\Z$ of groups. For $n\geq 3$, we can kill $\pi_1(SO(n)) =
\Z/2\Z$,
to get a double cover $\Spin(n)\to SO(n)$. The group $\Spin(n)$ is called the
\emph{spin group}. We have a cofiber sequence
$$B\Spin(n)\to BSO(n)\xar{w_2} K(\Z/2\Z,2).$$
If $w_2(\xi) = 0$, we get a further lift in the above diagram, begetting a
\emph{spin structure} on $\xi$.

Bott computed that $\pi_2(\Spin(n)) = 0$. However, $\pi_3(\Spin(n)) = \Z$;
killing this gives the \emph{string group} $\String(n)$. Unlike $\Spin(n)$,
$SO(n)$, and $O(n)$, this is not a finite-dimensional Lie group (since we have
an infinite dimensional summand $K(\Z,2)$). However, it can be realized as a
topological group. The resulting maps $$\String(n)\to \Spin(n)\to SO(n)\to
O(n)$$ are just the maps in the Whitehead tower for $O(n)$. Taking classifying
spaces, we get
$$
\xymatrix{
    & B\String(n)\ar[d] & \\
    & B\Spin(n)\ar[d] \ar[r]^{p_1/2} & K(\Z,4)\\
    & BSO(n)\ar[r]\ar[d] & K(\Z/2\Z,2)\\
    X\ar@{-->}[ur]\ar@{-->}[uur]\ar@{-->}[uuur]\ar[r]_{\xi} & BO(n)\ar[r]_{w_1}
    & B\Z/2\Z
}
$$

Computing the (mod $2$) cohomology of $BSO(n)$ is easy. We have a double cover
$BSO(n)\to BO(n)$ with fiber $S^0$. Consequently, there is a Gysin sequence:
$$
0\to H^q(BO(n)) \xar{w_1} H^{q+1}(BO(n)) \xar{\pi^\ast} H^{q+1}(BSO(n)) \to 0
$$
since $w_1$ is a nonzero divisor. The standard argument shows that
$$
H^\ast(BSO(n)) = \FF_2[w_2,\cdots,w_n].
$$
However, it is \emph{not} easy to compute $H^\ast(B\Spin(n))$ and
$H^\ast(B\String(n))$; these are extremely complicated (and only become more
complicated for higher connective covers of $BO(n)$). However, we will remark
that they are concentrated in even degrees.

To define integral characteristic classes for oriented bundles, we will need to
study Chern classes a little more.
%Suppose $V$ is a complex vector space.  We can then get an
%oriented real vector space $V_\RR$, with a choice of ordered basis given by
%$e_1,ie_1,\cdots,e_2,ie_2$. This space has an involution $-$, given by the opposite
%orientation. 
%
%The vector spaces $\overline{V}_\RR$ and $V_\RR$ are the same, but they have
%different orientations.
%%In fact, you have $\overline{V}_\RR = (-1)^{\dim V} V_\RR$.
%I can also consider $(V\otimes_\RR\cc)_\RR$.
%This is just $V\oplus V$, and we get that $(V\otimes_\RR\cc)_\RR = (-1)^{\dim V\cdot(\dim V - 1)/2} V\oplus V$.
%
%Another one.
%We know that $\overline{(V\otimes_\RR\cc)}\simeq V\otimes_\RR\cc$.
%
%We're going to use those identities.
Let $\xi$ be a complex $n$-plane bundle, and let $\overline{\xi}$ denote the
conjugate bundle. What is the total Chern class $c(\overline{\xi})$? Recall
that the Chern classes $c_k(\overline{\xi})$ occur as coefficients in the
identity
$$\sum c_i(\overline{\xi}) e(\lambda_{\overline{\xi}})^{n-i} = 0,$$
where $\lambda_{\overline{\xi}}\downarrow \PP(\overline{\xi})$. Note that
$\PP(\overline{\xi}) = \PP(\xi)$. By construction, $\lambda_{\overline{\xi}} =
\overline{\lambda_\xi}$. In particular, we find that
$$e(\lambda_{\overline{\xi}}) = -e(\lambda_\xi).$$
It follows that
$$
0 = \sum^n_{i=0}c_i(\overline{\xi})e(\overline{\lambda_\xi})^{n-i} =
\sum^n_{i=0}c_i(\overline{\xi}) (-1)^{n-i}e(\lambda_\xi)^{n-i} = (-1)^n
e(\lambda_\xi)^n + \cdots
$$
This is \emph{not} monic, and hence doesn't define the Chern classes of
$\overline{\xi}$. We do, however, get a monic polynomial by multiplying this
identity by $(-1)^n$:
$$
\sum^n_{i=0}(-1)^ic_i(\overline{\xi})e(\lambda_\xi)^{n-i} = 0.
$$
It follows that
$$
\boxed{c_i(\overline{\xi}) = (-1)^ic_i(\xi).}
$$
If $\xi$ is a real vector bundle, then
$$c_i(\xi\otimes\cC) = c_i(\overline{\xi\otimes\cC}) =
(-1)^ic_i(\xi\otimes\cC).$$
If $i$ is odd, then $2c_{i}(\xi\otimes\cC) = 0$. If $R$ is a $\Z[1/2]$-algebra,
we therefore define:
%We know what the cohomology of $BSO(n)$ is in $\FF_2$, and now I'm going to
%tell you what $H^\ast(BSO(n))$ is with coefficients in a $\Z[1/2]$-algebra (so
%$1/2$ exists). Then those odd Chern classes are just zero.
\begin{definition}
    Let $\xi$ be a real $n$-plane vector bundle. Then the $k$th Pontryagin
    class of $\xi$ is defined to be 
    $$p_k(\xi) = (-1)^kc_{2k}(\xi\otimes\cc)\in H^{4k}(X;R).$$
\end{definition}
Notice that this is $0$ if $2k>n$, since $\xi\otimes\cc$ is of complex
dimension $n$. The Whitney sum formula now says that:
$$
(-1)^k p_k(\xi\oplus\eta) = \sum_{i+j = k}(-1)^i p_i(\xi) (-1)^j p_j(\eta) =
(-1)^k\sum_{i+j=k}p_i(\xi)p_j(\eta).
$$
If $\xi$ is an oriented real $2k$-plane bundle, one can calculate that
$$
p_k(\xi) = e(\xi)^2\in H^{4k}(X;R).
$$
We can therefore write down the cohomology of $BSO(n)$ with coefficients in a
$\Z[1/2]$-algebra:
\begin{center}
    \begin{tabular}{ c|c c c c c c} 
	\hline
	$\ast = $ & 2 & 4 & 6 & 8 & 10 & 12\\
	\hline
	$H^\ast(BSO(2))$ & $e_2$ & $(e_2^2)$ & & & &\\
	$H^\ast(BSO(3))$ & & $p_1$ & & & &\\
	$H^\ast(BSO(4))$ & & $p_1,e_4$ & & $(e_4^2)$ & &\\
	$H^\ast(BSO(5))$ & & $p_1$ & & $p_2$ & &\\
	$H^\ast(BSO(6))$ & & $p_1$ & $e_6$ & $p_2$ & & $(e_6^2)$\\
	$H^\ast(BSO(7))$ & & $p_1$ & & $p_2$ & & $p_3$
    \end{tabular}
\end{center}
Here, $p_k \mapsto e_{2k}^2$. In the limiting case (i.e., for $BSO =
BSO(\infty)$), we get a polynomial algebra on the $p_i$.
\subsection{Applications}
We will not prove any of the statements in this section; it only serves as an
outlook. The first application is the following analogue of Theorem
\ref{thom-sw}:
\begin{theorem}[Wall]
    Let $M^n,N^n$ be oriented manifolds. If all Stiefel-Whitney numbers and
    Pontryagin numbers coincide, then $M$ is oriented cobordant to $N$, i.e.,
    there is an $(n+1)$-manifold $W^{n+1}$ such that
    $$\partial W^{n+1} = M\sqcup -N.$$
\end{theorem}
The most exciting application of Pontryagin classes is to Hirzebruch's
``signature theorem''. Let $M^{4k}$ be an oriented $4k$-manifold. Then, the
formula
$$x\otimes y\mapsto\langle x\cup y,[M]\rangle$$
defines a pairing
$$H^{2k}(M)/\mathrm{torsion}\otimes H^{2k}(M)/\mathrm{tors}\to \Z.$$
Poincar\'e duality implies that this is a perfect pairing, i.e., there is a
nonsingular symmetric bilinear form on $H^{2k}(M)/\mathrm{torsion}\otimes \RR$.
Every symmetric bilinear form on a real vector space can be diagonalized, so
that the associated matrix is diagonal, and the only nonzero entries are $\pm
1$. The number of $1$s minus the number of $-1$s is called the \emph{signature}
of the bilinear form. When the bilinear form comes from a $4k$-manifold as
above, this is called the signature of the manifold.
\begin{lemma}[Thom]
    The signature is an oriented bordism invariant.
\end{lemma}
This is an easy thing to prove using Lefschetz duality, which is a deep
theorem. Hirzebruch's signature theorem says:
\begin{theorem}[Hirzebruch signature theorem]
    There exists an explicit rational polynomial $L_k(p_1,\cdots,p_k)$ of
    degree $4k$ such that
    $$\langle L(p_1(\tau_M),\cdots,p_1(\tau_M)),[M]\rangle =
    \mathrm{signature}(M).$$
\end{theorem}
The reason the signature theorem is so interesting is that the polynomial
$L(p_1(\tau_M),\cdots,p_1(\tau_M))$ is defined only in terms of the tangent
bundle of the manifold, while the signature is defined only in terms of the
topology of the manifold. This result was vastly generalized by Atiyah and
Singer to the Atiyah-Singer index theorem.
\begin{example}
    One can show that 
    $$L_1(p_1) = p_1/3.$$
    The Hirzebruch signature theorem implies that $\langle
    p_1(\tau),[M^4]\rangle$ is divisible by $3$.
\end{example}
\begin{example}
    From Hirzebruch's characterization of the $L$-polynomial, we have
    $$L_2(p_1,p_2) = (7p_2 - p_1^2)/45.$$
    This imposes very interesting divisibility constraints on the
    characteristic classes of a tangent bundle of an $8$-manifold. This
    particular polynomial was used by Milnor to produce ``exotic spheres'',
    i.e., manifolds which are homeomorphic to $S^7$ but not diffeomorphic to
    it.
\end{example}
