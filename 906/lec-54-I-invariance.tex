\section{$I$-invariant of $\Bun_G$, and $G$-CW-complexes}
Everybody happy with principal bundles now?
Let $G$ be a topological group. Then $\Bun_G:\Top^{op}\to\Set$.
I want to show that this is $I$-invariant, i.e., given $X\xrightarrow{pr}X\times I$, we have $\Bun_G(X)\xrightarrow{\simeq}\Bun_G(X\times I)$.
This map is injective, for sure, because the composite $X\xrightarrow{in_0} X\times I\xrightarrow{pr}X$ gives you a splitting $\Bun_G(X)\xrightarrow{pr_\ast}\Bun_G(X\times I)\xrightarrow{in_0}\Bun_G(X)$ whose composite is the identity.

Suppose I have principal $G$-bundles $P\to X$ and $Q\to Y$, with a map $f:X\to Y$. Then I want to think about maps $P\to f^\ast Q$ over $X$. Equivalently, I want to think of $G$-equivariant dotted maps that make the following diagram commute.
\begin{equation*}
    \xymatrix{
	P\ar@{-->}[r]^g\ar[d] & Q\ar[d]\\
	X\ar[r]_f & Y
    }
\end{equation*}
Suppose I have $P\to X\times I$; then we get $in_0^\ast P\to X$. This has to be what you get when you ---. All we have to do is construct a map $P\to in_0^\ast P$ like in the diagram above.

We'll do this for $X$ a CW-complex. I'll refer you to Husemoller for a different argument -- he does the general case.
\subsection{$G$-CW-complexes}
What's a ``$G$-cell''? The answer is $D^n\times H\backslash G$, for closed subgroups $H<G$. The space $H\backslash G$ is a right $G$-space. It's an orbit.
The boundary of $D^n\times H\backslash G$ is just $\partial D^n\times H\backslash G$. We can build up $G$-CW-complexes exactly as we build up CW-complexes.
\begin{definition}
    A $G$-CW-complex is a (right) $G$-space $X$ with a filtration $0=X_{-1}\subseteq X_0\subseteq \cdots\subseteq X$ such that for all $n$, there exists a pushout square:
    \begin{equation*}
	\xymatrix{
	    \coprod\partial D^n_\alpha\times H_\alpha\backslash G\ar[r]\ar[d] & \coprod D^n_\alpha\times H_\alpha\backslash G\ar[d]\\
	    X_{n-1}\ar[r] & X_n
	    }
    \end{equation*}
    and $X$ has the direct limit topology.
\end{definition}
So a CW-complex is a $G$-CW-complex for the trivial group $G$.
\begin{theorem}
    If $G$ is a compact Lie group and $M$ a compact smooth $G$-manifold. Then $M$ admits a $G$-CW-structure.
\end{theorem}
This is a hard theorem.
Note that if $G$ acts principally\footnote{The action that you have when you say you have a principal $G$-bundle.} on $P$, then every $G$-CW-structure on $P$ is ``free'', i.e., $H_\alpha = 0$.
\begin{enumerate}
    \item If $X$ is a $G$-CW-complex, then $X/G$ inherits a CW-structure where $(X/G)_n = X_n/G$.
    \item If $P\to X$ is principal, then a CW-structure on $X$ lifts to a $G$-CW-structure on $P$.
\end{enumerate}
This equivariant stuff is very popular these days.
\subsection{Proof of $I$-invariance}
First thing to notice is that if $X\simeq\ast$, then any principal $G$-bundle over $X$ is trivial, i.e., $P\simeq X\times G$ as a $G$-bundle. That iso isn't unique. How do we prove this? This is a case of the general theorem.
Here's the thing to notice. If $P\downarrow X$ has a section, then it's trivial. That's worth contemplating. Why is that?
Suppose I have a section $s:X\to P$. This $P$ has an action of the group on it, so I can extend this to $X\times G\to P$ via $(x,g)\mapsto gs(x)$.
This is a map of $G$-bundles over $X$, so it's an isomorphism.
So, to show the general result, we have to construct a section. Let's take any map $X\to P$. Say the constant map!
Then the following diagram commutes up to homotopy, and hence there's an \emph{actual} section of $P\to X$, as desired.
\begin{equation*}
    \xymatrix{
	& P\ar[d]\\
	X\ar[ur]^{const}\ar[r] & X
    }
\end{equation*}
OK, for the general case, let's assume $X$ is a CW-complex. For notational convenience, let's write $Y=X\times I$.
Filter $Y$ by subcomplexes: define $Y_0 = X\times 0$.
In general, define $Y_n = X\times 0\cup X_{n-1}\times I$.
Thus, I can construct $Y_n$ out of $Y_{n-1}$ via a pushout:
\begin{equation*}
    \xymatrix{
	\coprod_{\alpha\in\Sigma_{n-1}}(\partial D^{n-1}\times I \cup D^{n-1}_\alpha\times 0) \ar[r]\ar[d]_{\coprod_{\alpha\in\Sigma_{n-1}} f_\alpha\times 1_I\cup\phi_\alpha\times 0} & \coprod_\alpha(D^{n-1}_\alpha\times I) \ar[d]\\
	Y_{n-1}\ar[r] & Y_n
    }
\end{equation*}
where the maps $f_\alpha$ and $\psi_\alpha$ are defined as:
\begin{equation*}
    \xymatrix{
	\partial D^{n-1}_\alpha\ar[r]^{f_\alpha}\ar[d] & X_{n-2}\ar[d]\\
	D^{n-1}_\alpha\ar[r]_{\phi_\alpha} & X_{n-1}
    }
\end{equation*}
Remember I have $P\xrightarrow{p}Y = X\times I$. Define $P_n = p^{-1}(Y_n)$.
We can build $P_n$ from $P_{n-1}$ in a similar way:
\begin{equation*}
    \xymatrix{
	\coprod_\alpha(\partial D^{n-1}_\alpha\times I\cup D^{n-1}_\alpha\times 0)\times G\ar[r]\ar[d] & \coprod_\alpha (D^{n-1}_\alpha\times I)\times G\ar[d]\\
	P_{n-1}\ar[r] & P_n
    }
\end{equation*}
This makes sense since $D^n_\alpha\times I$ is a contractible space, and same thing for the other factor.
Note that this isn't \emph{quite} a $G$-CW-structure.
Remember that what I'm trying to do is fill in a dotted map:
\begin{equation*}
    \xymatrix{
	P\ar[r]\ar[d] & P_0\ar[d]\\
	Y\ar[r]_{pr} & P_0
    }
\end{equation*}
I'm constructing this inductively-- we have $P_{n-1}\to P_0$. So I want to define $\coprod_\alpha(D^{n-1}_\alpha\times I)\times G\to P_0$ that's equivariant. That's the same thing as a map $\coprod_\alpha (D^{n-1}_\alpha\times I)\to P_0$ that's compatible with the map from $\coprod(\partial D^{n-1}_\alpha\times I\cup D^{n-1}_\alpha\times 0)$. Namely, I want to fill in:
\begin{equation}\label{finally}
    \xymatrix{
	\coprod_\alpha (\partial D^{n-1}_\alpha\times I\cup D^{n-1}_\alpha\times 0)\ar[r]\ar[d] & \coprod_\alpha(D^{n-1}_\alpha\times I)\ar@{-->}[dddr]\ar[d] & \\
	\coprod_\alpha(\partial D^{n-1}_\alpha\times I\cup D^{n-1}_\alpha\times 0)\times G\ar[r]\ar[d] & \coprod_\alpha (D^{n-1}_\alpha\times I)\times G\ar@{-->}[ddr]\ar[d] & \\
	P_{n-1}\ar[drr]_{\mathrm{induction}}\ar[r] & P_n\ar[dr] & \\
	& & P_0\ar[d]\\
	& & X
    }
\end{equation}
Now, I know that $(D^{n-1}\times I,\partial D^{n-1}\times I\cup D^{n-1}\times 0)\simeq (D^{n-1}\times I,D^{n-1}\times 0)$. So what I have is:
\begin{equation*}
    \xymatrix{
	D^{n-1}\times 0\ar[d]\ar[r]^{\mathrm{induction}} & P_0\ar[d]\\
	D^{n-1}\times I\ar[r]_{\phi\circ pr}\ar@{-->}[ur] & X
    }
\end{equation*}
So the dotted map exists, since $P_0\to X$ is a fibration!

OK, so note that I haven't checked that the outer diagram in Equation \ref{finally} commutes, because otherwise we wouldn't get $P_n\to P_0$.
\begin{exercise}
    Check my question above.
    
    Turns out this is easy, because you have a factorization:
\begin{equation*}
    \xymatrix{
	D^{n-1}\times 0\ar[d]\ar[r] & P_{n-1}\ar[r]^{\mathrm{induction}} & P_0\ar[d]\\
	D^{n-1}\times I\ar[rr]_{\phi\circ pr}\ar@{-->}[urr] & & X
    }
\end{equation*}
\end{exercise}
Oh my god, look what time it is! Oh well, at least we got the proof done.
