\section{Classifying spaces of groups}\label{classifying-g-bundles}
%I learned this attitude towards classifying spaces from Segal.
%An article is on the website.
%
The constructions of the previous sections can be summarized in a single
diagram:
\begin{equation*}
    \xymatrix{
	\Cat\ar[r]^{\text{nerve}} & s\Set\ar[d]^{|-|}\\
	\mathbf{Gp}\ar@{^(->}[u]\ar[r]_{B} & \Top
    }
\end{equation*}
The bottom functor is defined as the composite along the outer edge of the
diagram. The space $BG$ for a group $G$ is called the \emph{classifying space
of $G$}. At this point, it is far from clear what $BG$ is classifying. The goal
of the next few sections is to demystify this definition.

\begin{lemma}\label{conjugation-homotopic}
    Let $G$ be a group, and $g\in G$.  Let $c_g:G\to G$ via $x\mapsto
    gxg^{-1}$.  Then the map $Bc_g:BG\to BG$ is homotopic to the identity.
\end{lemma}
\begin{proof}
    The homomorphism $c_g$ is a functor from $G$ to itself. It suffices to
    prove that there is a natural transformation $\theta$ from the identity to
    $c_g$. This is rather easy to define: it sends the only object to the only
    object: we define $\theta_\ast:\ast\to \ast$ to be the map given by
    $\ast\xrightarrow{g}\ast$ specified by $g\in \Hom_G(\ast,\ast) = G$. In
    order for $\theta$ to be a natural transformation, we need the following
    diagram to commute, which it obviously does:
    \begin{equation*}
	\xymatrix{
	    \ast\ar[d]_1\ar[r]^g & \ast\ar[d]^{gxg^{-1}}\\
	    \ast\ar[r]_{g} & \ast.
	    }
    \end{equation*}
\end{proof}
Groups are famous for acting on objects. Viewing groups as categories allows
for an abstract definition a group action on a set: it is a functor $G\to
\Set$. More generally, if $\cc$ is a category, an action of $\cc$ is a functor
$\cc\xrightarrow{X}\Set$. We write $X_c = X(c)$ for an object $c$ of $\cc$.
\begin{definition}
    The ``translation'' category $X\cc$ has objects given by
    $$\mathrm{ob}(X\cc) = \coprod_{c\in \cc}X_c,$$
    and morphisms defined via $\Hom_{X\cc}(x\in X_c, y\in X_d) = \{f:c\to
    d:f_\ast(x) = y\}$.
\end{definition}
There is a projection $X\cc\to \cc$. (For those in the know: this is a special
case of the Grothendieck construction.)
\begin{example}
    The group $G$ acts on itself by left translation. We will write
    $\widetilde{G}$ for this $G$-set. The translation category $\widetilde{G}G$
    has objects as $G$, and maps $x\to y$ are elements $yx^{-1}$. This category
    is ``unicursal'', in the sense that there is exactly one map from one
    object to another object. Every object is therefore initial and terminal,
    so the classifying space of this category is trivial by the discussion at
    the end of \S \ref{classifying-space-properties}. We will denote by $EG$
    the classifying space $B(\widetilde{G}G)$. The map $\widetilde{G}G\to G$
    begets a canonical map $EG\to BG$.
\end{example}
The $G$ also acts on itself by right translation. Because of associativity, the
right and left actions commute with each other. It follows that the right
action is equivariant with respect to the left action, so we get a right action
of $G$ on $EG$.
\begin{claim}
    This action of $G$ on $EG$ is a principal action, and the orbit projection
    is $EG\to BG$.
\end{claim}
To prove this, let us contemplate the set $N(\widetilde{G}G)_n$. An element is
a chain of composable morphisms. In this case, it is actually just a sequence
of $n+1$ elements in $G$, i.e., $N(\widetilde{G}G)_n = G^{n+1}$. The right
action of $G$ is simply the diagonal action. We claim that this is a free
action. More precisely:
\begin{lemma}[Shearing]
    If $G$ is a group and $X$ is a $G$-set, and if $X\times^\Delta G$ has the
    diagonal $G$-action and $X\times G$ has $G$ acting on the second factor by
    right translation, then $X\times^\Delta G\simeq X\times G$ as $G$-sets.
\end{lemma}
\begin{proof}
    Define a bijection $X\times^\Delta G\mapsto X\times G$ via $(x,g)\mapsto
    (xg^{-1},g)$. This map is equivariant since $(x,g)\cdot h = (xh, gh)$,
    while $(xg^{-1}, g) \cdot h = (xg^{-1}, gh)$. The element $(xh, gh)$ is
    sent to $(xh(gh)^{-1}, gh)$, as desired.
    The inverse map $X\times G\to X\times^\Delta G$ is given by $(x,g)\mapsto
    (xg, g)$.
\end{proof}
We know that $G$ acts freely on $N(\widetilde{G}G)_n$, soo a nonidentity group
element is always going to send a simplex to another simplex. It follows that
$G$ acts freely on $EG$.

To prove the claim, we need to understand the orbit space. The shearing lemma
shows that quotienting out by the action of $G$ simply cancels out one copy of
$G$ from the product $N(\widetilde{G}G) = G^n$. In symbols:
$$N(\widetilde{G} G)/G\simeq G^n\simeq (NG)_n.$$
Of course, it remains to check the compatibility with the face and degeneracy
maps. We will not do this here; but one can verify that everything works out:
the realization is just $BG$!

We need to be careful: the arguments above establish that $EG/G \simeq BG$ when
$G$ is a finite group. The case when $G$ is a topological group is more
complicated. To describe this generalization, we need a preliminary categorical
definition.

Let $\cc$ be a category, with objects $\cc_0$ and morphisms $\cc_1$. Then we
have maps $\cc_1\times_{\cc_0}\cc_1\xrightarrow{\text{compose}}\cc_1$ and two
maps (source and target) $\cc_1\to \cc_0$, and the identity $\cc_0\to\cc_1$.
One can specify the same data in any category $\cd$ with pullbacks. Our
interest will be in the case $\cd = \Top$; in this case, we call $\cc$ a
``category in $\Top$''.

%The story described above works just as well if we were to consider functors
%$X:\cc\to \Top$ instead of $X:\cc\to \Set$. Let $G$ be a topological group. Define $\widetilde{G}$ to be a category in $\Top$.
%
%An example is a topological group -- that's a category in $\Top$.
Let $G$ be a topological group acting on a space $X$. We can again define $XG$,
although it is now a category in $\Top$. Explicitly, $(XG)_0 = X$ and $(XG)_1 =
G\times X$ as spaces. The nerve of a topological category begets a simplicial
space. In general, we will have
$$(N\cc)_n = \cc_1\times_{\cc_0}\cc_1\times\cdots\times_{\cc_0}\cc_1.$$
The geometric realization functor works in exactly the same way, so the
realization of a simplicial space gets a topological space.
%
%You need to put some mild topological conditions on the group for the theorem we proved above to be true.
%Again, we have $\widetilde{G}$ with $G$ acting on it.
%Again, we let $EG = B\widetilde{G}G$.
The above discussion passes through with some mild topological conditions on
$G$ (namely, if $G$ is an absolute neighborhood retract of a Lie group); we
conclude:
\begin{theorem}
    Let $G$ be an absolute neighborhood retract of a Lie group. Then $EG$ is
    contractible, and $G$ acts from the right principally. Moreover, the map
    $EG\to BG$ is the orbit projection.
\end{theorem}
A generalization of this result is:
\begin{exercise}
    Let $X$ be a $G$-set. Show that
    $$EG\times_G X \simeq B(XG).$$
\end{exercise}
