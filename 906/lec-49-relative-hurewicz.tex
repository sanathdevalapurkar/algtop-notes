\section{Relative Hurewicz and J.~H.~C.~Whitehead}
I find what I'm talking about today very confusing, because there's this number, this $n$. I'm going to essentially read from my notes, because I'll get confused otherwise.
\begin{definition}
    Let $n\geq 0$. Say that $X$ is $(n-1)$-connected if for all $0\leq k\leq n$, and any $f:S^{k-1}\to X$ extends:
    \begin{equation*}
	\xymatrix{
	S^{k-1}\ar[d]\ar[r] & X\\
	D^k\ar@{-->}[ur] & 
	}
    \end{equation*}
\end{definition}
    Let's see what happens in low dimensions. When $n=0$, we know that $S^{-1} = \emptyset$, and $D^0 = \ast$. Saying that you're $(-1)$-connected is just saying that you're nonempty. If $n=1$, then this is saying that you're path connected, i.e., $0$-connected iff path connected. You can check that this is exactly the same as what we said before, using homotopy groups.

\begin{definition}
    Let $n\geq 0$. Say that $(X,A)$ is $n$-connected if for all $0\leq k\leq n$, there's a lift:
    \begin{equation*}
	\xymatrix{
	    (D^k,S^{k-1})\ar[r]^f\ar@{-->}[d] & (X,A)\\
	    (A,A)\ar[ur] & 
	    }
    \end{equation*}
    up to homotopy. In other words, you can homotope $f$ to a map with image in $A$, without moving $f|_{S^{k-1}}$.
\end{definition}
What does this mean for small values? When $n=0$, we find that $0$-connected means that $A$ meets every path component of $X$.

Equivalently, we can state:
    \begin{definition}
	$(X,A)$ is $n$-connected if:
	\begin{itemize}
	    \item $n=0$: then $\pi_0(A)\to \pi_0(X)$ surjects.
	    \item $n>0$: then $\pi_0(A)\xrightarrow{\simeq}\pi_0(X)$, and for all $a\in A$, $\pi_k(X,A,a) = 0$ for $1\leq k\leq n$. Equivalently, $\pi_0(A)\xrightarrow{\simeq}\pi_0(X)$ and $\pi_k(A,a)\to\pi_k(X,A)$ is an isomorphism for $1\leq k<n$ and is onto for $k=n$.
	\end{itemize}
    \end{definition}
    So ``$\pi_0(X,A) = 0$'' makes sense, because it means $\pi_0(A)\to \pi_0(X)$ surjects. I can't define $\pi_0(X,A)$, but you should think of it this way.

Let's try to write down what the relative Hurewicz theorem might be.
\subsection{Relative Hurewicz}
    Assume that $\pi_0(A) = \ast = \pi_0(X)$, and pick $a\in A$, which I won't bother about any more. We have:
    \begin{equation*}
	\xymatrix{
	    \cdots\ar[r] & \pi_2(X,A)\ar[r]\ar[d]^h & \pi_1(A)\ar[r]\ar[d]^h & \pi_1(X)\ar[r]\ar[d]^h & \pi_1(X,A)\ar[r]\ar[d]^h & \pi_0(A)\ar[r]\ar[d]^h & \pi_0(X)\ar[d]^h & \\
	    \cdots\ar[r] & H_2(X,A)\ar[r] & H_1(A)\ar[r] & H_1(X)\ar[r] & H_1(X,A)\ar[r] & H_0(A)\ar[r] & H_0(X)\ar[r] & H_0(X,A)
	    }
    \end{equation*}
How do we define the relative Hurewicz map? Let $\alpha\in \pi_n(X,A)$; then $\alpha:(D^n,S^{n-1})\to (X,A)$. So we pick a generator in $\Z\in H_n(D^n,S^{n-1})$, and send it to something in $H_n(X,A)$ under the induced $\alpha^\ast:H_n(D^n,S^{n-1})\to H_n(X,A)$.

Because $H_n(X,A)$ is abelian, we find that $\pi_1(A)$ acts trivially on $H_n(X,A)$, i.e., $h(\omega(\alpha)) = h(\alpha)$. Thus the relative Hurewicz map factors through $\pi_n^\dagger(X,A)$, which is $\pi_n(X,A)$ quotiented out by the normal subgroup generated by $(\omega\alpha)\alpha^{-1}$ where $\omega\in\pi_1(A)$ and $\alpha\in \pi_n(X,A)$. Thus you get a map $\pi_n^\dagger(X,A)\to H_n(X,A)$, that's more likely to be an isomorphism.
\begin{theorem}[Relative Hurewicz]
    Let $n\geq 1$. Assume $(X,A)$ is $n$-connected. Then $H_k(X,A) = 0$ for $0\leq k\leq n$, and $\pi_{n+1}^\dagger(X,A)\to H_{n+1}(X,A)$ is an isomorphism.
\end{theorem}
\begin{proof}
I won't prove this now; with any luck, we'll come back and prove this once we have the Serre spectral sequence in our toolkit.
\end{proof}
This is an extremely useful result. Now, that's it! That's all I wanted to say about the relative Hurewicz theorem. Let's apply this to get surprising results.
\subsection{J.~H.~C.~Whitehead theorems}
He was an interesting character. He raised pigs. He's interested in a continuous map $f:X\to Y$, and when that's an isomorphism in homology or homotopy, and relate them.
\begin{definition}
    Let $f:X\to Y$ and $n\geq 0$. Say that $f$ is an $n$-equivalence (some would say $n$-connected, but I won't say that here) if for every $\ast\in Y$, the homotopy fiber $F(f,\ast)$ is $(n-1)$-connected.
\end{definition}
A $0$-equivalence means that $\pi_0(X)$ surjects onto $\pi_0(Y)$. For $n>0$, this is saying that $\pi_0(X)\xrightarrow{\simeq}\pi_0(Y)$, and for every $\ast\in X$:
\begin{equation*}
    \pi_k(X,\ast)\to\pi_k(Y,f(\ast)) \text{ is }\begin{cases}
	\text{iso } & 1\leq k<n\\
	\text{epi, i.e., onto } & k = n
    \end{cases}
\end{equation*}
Would you like me to tell you about how to replace a map by a cofibration? Never mind -- this is the first problem! So, using the ``mapping cylinder'', we can always assume $f:X\hookrightarrow Y$, and then $f:X\to Y$ is an $n$-equivalence iff $(Y,X)$ is $n$-connected.
\begin{theorem}[J.~H.~C.~Whitehead 1]
    Suppose $n\geq 0$, and $f:X\to Y$ is $n$-connected. Then:
\begin{equation*}
H_k(X)\to H_k(Y) \text{ is }\begin{cases}
\text{iso } & 1\leq k<n\\
\text{epi, i.e., onto } & k = n
\end{cases}
\end{equation*}
\end{theorem}
\begin{proof}
    When $n=0$, because $\pi_0(X)\to \pi_0(Y)$ is surjective, $H_0(X)\simeq \Z[\pi_0(X)]\to \Z[\pi_0(Y)]\simeq H_0(Y)$ is surjective. The relative Hurewicz theorem gives the rest. Note that the relative Hurewicz dealt with $\pi_n^\dagger(X,A)$, but $\pi_n(X,A)\to\pi_n^\dagger(X,A)$ is surjective.
\end{proof}
Let me call out what happens when $n=\infty$.
\begin{definition}
    $f$ is a \emph{weak equivalence} (or an $\infty$-equivalence, to make it sound more impressive) if it's an $n$-equivalence for all $n$, i.e., it's a $\pi_\ast$-isomorphism.
\end{definition}
Putting everything together, we obtain:
\begin{corollary}
    A weak equivalence induces an isomorphism in integral homology.
\end{corollary}
Let's try to go in the other direction.

If $H_0(X)\to H_0(Y)$ surjects, then $\pi_0(X)\to \pi_0(Y)$ surjects also. Good start! Assume $X$ and $Y$ path connected. Suppose $H_1(X)$ surjects onto $H_1(Y)$. I'd like to know that $\pi_1(X)\to\pi_1(Y)$ surjects. It's hard to know this, though, because $H_1(X)$ is the abelianization. Let's give up and assume that $\pi_1(X)\to \pi_1(Y)$ is surjective.

Assume $\pi_1(X)$ surjects onto $\pi_1(Y)$. Assume $H_2(X)\to H_2(Y)$ surjects, and $H_1(X)\xrightarrow{\simeq}H_1(Y)$ surjects. We know that $H_2(Y,X) = 0$. On the level of the Hurewicz maps, I still don't know what to do because I only know about $\pi_2^\dagger$. What to do? Let's assume that $\pi_1(X)$ is trivial. That's a pretty radical assumption. It'd be enough to ask that $\pi_1(X)$ acts trivially on $\pi_2(Y,X)$. That's \emph{technically} what you want to assume. That's basically impossible to check. Anyway, if $\pi_1(X) = 0$, then $\pi_1(Y) = 0$. So $\pi_2(Y,X)$ is trivial, and we can go up the ladder.
\begin{theorem}[J.~H.~C.~Whitehead 2]
    Let $n\geq 2$. Assume that $\pi_1(X) = 0 = \pi_1(Y)$, and I have $f:X\to Y$ such that:
\begin{equation*}
H_k(X)\to H_k(Y) \text{ is }\begin{cases}
\text{iso } & 1\leq k<n\\
\text{epi, i.e., onto } & k = n
\end{cases}
\end{equation*}
then $f$ is an $n$-equivalence.
\end{theorem}
\begin{corollary}
    Let $X$ and $Y$ be simply-connected. If $f$ induces an isomorphism in homology, then it's a weak equivalence.
\end{corollary}
That's an \emph{incredibly} useful theorem, because we can actually compute homology! Let's complete this with a further theorem we'll prove on Wednesday.
\begin{theorem}
    If $Y$ is a CW-complex, a weak equivalence $f:X\to Y$ is in fact a homotopy equivalence!
\end{theorem}
