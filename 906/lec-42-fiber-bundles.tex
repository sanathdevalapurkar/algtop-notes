\section{Fiber bundles, fibrations, cofibrations}
Having set up the requisite technical background, 
we can finally launch ourselves from point-set topology to the world of homotopy theory.
\subsection{Fiber bundles}
\begin{definition}\label{fiberbundle}
    A fiber\footnote{Or ``fibre'', if you're British.} bundle is a map $p:E\to B$,
    such that for every $b\in B$, there exists
    \begin{itemize}
	\item an open subset $U\subseteq B$ that contains $b$, and
	\item a map $p^{-1}(U)\to p^{-1}(b)$ such that $p^{-1}(U)\to U\times p^{-1}(b)$ is a homeomorphism.
    \end{itemize}
    If $p:E\to B$ is a fiber bundle, $E$ is called the \emph{total space}, $B$ is called the
    \emph{base space}, $p$ is called a \emph{projection},
    and $F$ (sometimes denoted $p^{-1}(b)$) is called the \emph{fiber over $b$}.
\end{definition}
In simpler terms: the preimage over every point in $B$ looks like a product,
i.e., the map $p:E\to B$ is ``locally trivial'' in the base.

Here is an equivalent way of stating Definition \ref{fiberbundle}:
there is an open cover $\cU$ (called the \emph{trivializing cover}) of $B$,
such that for every $U\subseteq \cU$,
there is a space $F$, and a homeomorphism
$p^{-1}(U)\simeq U\times F$ that is compatible with the projections down to $U$.

Fiber bundles are naturally occuring objects.
For instance, a covering space $E\to B$ is a fiber bundle with discrete fibers. 
\begin{example}[The Hopf fibration]
    The Hopf fibration is an extremely important example of a fiber bundle.
    Let $S^3\subset \cC^2$ be the $3$-sphere.
    There is a map $S^3\to \CP^1 \simeq S^2$ that is given by sending a vector $v$ to the
    complex line through $v$ and the origin.
    This is a non-nullhomotopic map, and is a fiber bundle whose fiber is $S^1$.
\end{example}
The Hopf fibration is a map between smooth manifolds.
A theorem of Ehresmann's says that it is not too hard to construct fiber bundles over smooth manifolds:
\begin{theorem}[Ehresmann]
    Suppose $E$ and $B$ are smooth manifolds, and let $p:E\to B$ be a smooth (i.e., $C^\infty$) map.
    Then $p$ is a fiber bundle if:
    \begin{enumerate}
	\item $p$ is a \emph{submersion}, i.e., $dp:T_e E\to T_{p(e)} B$ is a surjection, and
	\item $p$ is \emph{proper}, i.e., preimages of compact sets are compact.
    \end{enumerate}
\end{theorem}
%For example, I can look at the complement of a closed set in $S^3$, and then the restriction of $p$ won't be a fiber again.
The purpose of this part is really about understanding fiber bundles via algebraic
methods like cohomology and homotopy.
We will usually need some ``niceness'' condition when studying fiber bundles;
this is made precise in the following definition (see \cite{MayConcise}).
\begin{definition}
    Let $X$ be a space.
    An open cover $\cU$ of $X$ is said to be \emph{numerable}
    if there exists a subordinate partition of unity, i.e.,
    for each $U\in\cU$, there is a function $f_U:X\to [0,1]=I$
    such that $f^{-1}((0,1]) = U$,
    and any $x\in X$ belonds to only finitely many $U\in\cU$.
    The space $X$ is said to be \emph{paracompact} if any open cover admits a numerable refinement.
\end{definition}
This isn't too restrictive for us algebraic topologists since CW-complexes are paracompact.
\begin{definition}
    A fiber bundle is said to be \emph{numerable} if it admits a numerable trivializing cover.
\end{definition}

\subsection{Fibrations}
For our purposes, though, fiber bundles are still too narrow.
Fibrations capture the essence of fiber bundles.
%Edit from here.
\begin{definition}
    A map $p:E\to B$ is called a \emph{Hurewicz fibration} (I'll just say fibration) if it satisfies the homotopy lifting property (HLP). That says this. Suppose I have a homotopy $h:I\times W\to B$. Then there exists a lift:
    \begin{equation*}
	\xymatrix{
	    W\ar[r]^f\ar[d]_{\mathrm{in}_0} & E\ar[d]^p\\
	    I\times W\ar[r]_h\ar@{-->}[ur]^{\overline{h}} & B
	    }
    \end{equation*}
    Where the outer diagram commutes. It doesn't need to be a unique lift!
\end{definition}
This is an extremely alarming definition. This has to be checked for all spaces and all maps and all homotopies! So it took an act of genius to construct something like this. The idea of doing something like this goes back to Serre back in about 1950. It solved a problem (of defining what a fibration was).

But it's not impossible to check! Let me tell you some stuff. Hurewicz was a faculty member here, one of the first algebraic topologists here. I'm going to try to reformulate this diagram here a little bit. Let me try to adjoint the $I$. Then I have:
\begin{equation*}
    \xymatrix{
	E\ar[r]^p & B\\
	W\ar[u]^f\ar[r]_{\widehat{h}} & B^I\ar[u]_{\mathrm{ev}_0}
    }
\end{equation*}
This is just the adjoint of our diagram. Now, a diagram like this is the same as a map $W\to B^I\times_B E$ (the pullback is the set of paths and points $(\omega, e)$ such that $\omega(0) = p(e)$. But now, if our dotted map exists, we'd have a lifted homotopy $\widehat{\overline{h}}:W\to E^I$. We have a map $\widetilde{p}:E^I\to B^I\times_B E$ given by $\omega\mapsto (p\omega,\omega(0))$. Clearly $p\omega(0) = p\omega(0)$, so this lands in $B^I\times_B E$.

Thus the existence of $\overline{h}$ is the same as a lift:
\begin{equation*}
    \xymatrix{
	& E^I\ar[d]^{\widetilde{p}}\\
	W\ar[r]\ar@{-->}[ur]^{\widehat{\overline{h}}} & B^I\times_B E
    }
\end{equation*}
Obviously the universal example is $B^I\times_B E$. If $p$ is a fibration, then I can make the lift in the following diagram, and if I can lift for any $W$, I can obviously construct the lift in the following diagram:
\begin{equation*}
    \xymatrix{
	& E^I\ar[d]^{\widetilde{p}}\\
	B^I\times_B E\ar@{-->}[ur]^\lambda\ar[r]^1 & B^I\times_B E
    }
\end{equation*}
We say $\lambda$ is a \emph{lifting function}. Well, if $\omega(0) = p(e)$, then $\lambda(\omega:I\to B, e\in E):I\to E$. And in particular, $p\circ\lambda(\omega, e) = \omega$ and $\lambda(\omega,e)(0) = e$. So $\lambda$ starts with a path $\omega$ in $B$, and some point over the starting point, and produces a path in $E$ which lives over $\omega$. In other words, it's a path lifting. The key thing is that it's a continuous way to lift. So you \emph{can} check the HLP in certain cases.
\begin{theorem}[Dold]
    Let $p:E\to B$ be a map. Assume there's a numerable cover of $B$, say $\cU$, such that for every $U\in\cU$, the restriction $p|_{p^{-1}(U)}:p^{-1}U\to U$ is a fibration. (It's locally a fibration over the base). Then $p$ itself is a fibration.
\end{theorem}
Check for yourself that at least a product projection $\mathrm{pr}_1:B\times F\to B$ is a product fibration (e.g. using the universal property or something). So in particular, \emph{every numerable fiber bundle is a fibration}. I.e., numerable fiber bundles satisfy the homotopy lifting property. We'll see why this is such a good thing next week. Questions?
